\documentclass[./main_en.tex]{subfiles}
\graphicspath{{\subfix{./figs}}}

% ------------ main document ------------
\begin{document}
\doublespacing % line spacing
\large

% style setup
\newpage
\renewcommand{\headrulewidth}{0pt}
\thispagestyle{fancy}
%... then configure it
\fancyhf{} % Clear all header and footer fields.
\fancyfoot{} % clear all footer fields
\fancyfoot[C]{\thepage}

\begin{center}
    \sffamily{\Large \textbf{Agradecimentos}}
    \vspace{5mm}
\end{center}
\singlespacing
\normalsize

% custom paragraph skip
\setlength{\parskip}{\myparskip}

\par I am grateful to my professor and advisor, Guilherme Fernandes Marques, for the teachings, encouragement, partnership, and, above all, for the trust placed in me since 2018. This is a journey of seven years. We are already seeing some fruits ripen. But it’s just the beginning.

\par I thank my academic colleagues at \texttt{GESPLA}, the research group on water resources planning and management at \texttt{IPH}. For the lunches and dinners at the University Restaurant, for the jokes and laughter, for the company in focused work afternoons. Among the old and new guards I had the opportunity to work with, my thanks go especially, but not exclusively, to Ana Paula Dalcin, Ariane Sigallis, Gláucio Souza, Júlia Daiello, Giúlia Carrard, Juliano Finck, Julieta Nhampossa, Luísa Lucchese, Márcio Inada, Paola Kuele, Rossano Belladona.

\par I am grateful to my colleagues in the \texttt{IPH} graduate program who tolerated me when I barged into rooms, asking what they were doing and how much they truly believed in what they were doing. To those responsible for the Water Science Student Seminar, who elected me the wildcard speaker for all editions. With special affection, I thank the colleagues who created the "Brothers' House" in Aracaju: Bruno Abatti, Carlos Ferrari, Cléber Gama, Juliana Andrade, Lara Nonnemacher, Marina Fagundes, and Priscila Kipper (here it is, finally, the IpoTese). I am also immensely grateful to colleague and friend Rafael Barbedo, \textit{O Veredas}, for the partnership and for sharing his knowledge about \texttt{HAND} during bohemian nights in Vienna, Berlin, and Bom Fim. To the brilliant colleagues from \say{USP} (that is, not from \texttt{IPH}), I thank them for the friendship and excellent exchanges we had in Vienna, Belo Horizonte, and Aracaju: Dimaghi Schwamback, Julien Sone, and Rodrigo Perdigão.

\par I thank the network of mentors who guided me before graduate school. At 19, in 2011, I began scientific initiation, helping the now Professor Maria Cristina de Almeida to prepare bio-hydrogen experiments. At that time, the guidance of the late Professor Olinto Monteggia was a remarkable inspiration for inventiveness and curiosity. Around that period, I also interacted with Ayan Fleischmann, whom I also thank for fruitful exchanges during graduate school. I thank the late Professor Dieter Wartchow for the inspiration that led me to pursue socially inclusive engineering. I thank Professor Rualdo Menegat for inspiring me, especially with the "Environmental Atlas of Porto Alegre," an encyclopedic work spectacularly illustrated. Motivated by the Atlas, I attempt to summon the spirit of Humboldt and Lovelock here. I thank Professor Tatiana Silva, and colleagues from \texttt{LABMODEL} and \texttt{FURG}, for practical experience beyond the University -- and for transforming me from a future microbiologist into the \say{map guy}. I thank my last advisor, Professor Fernando Dornelles, who brought me back to \texttt{IPH} at the end of my undergraduate studies, in 2016.

\par I also feel the duty to thank the citizens of the Federative Republic of Brazil, who, through taxes, support the academic infrastructure that I was fortunate to enjoy and, in various projects, was funded by. Thus, I thank the National Water and Basic Sanitation Agency, the Municipality of São Leopoldo, and the Riograndense Sanitation Company for granting me scholarships during the doctoral period. In particular, I am retroactively grateful to the government of President Dilma Rousseff, who in 2014 created an academic exchange program, Science Without Borders, now extinct. Without the luck of studying in Philadelphia, in the United States, my struggles with English would have made diving into the literature prohibitive for this thesis. Even in global quarantine, without leaving home, I could explore knowledge thanks to the internet, the \texttt{CAPES} system, and my command of the English language.

\par To the professors who somehow accompanied me during my doctorate, my thanks go to Professors Walter Collischonn and Rodrigo Paiva, who were on my qualification board, and to Professor Anderson Ruhoff, who showed me the way to make real evapotranspiration estimates. I also thank Professor Nilza de Castro for trusting me to create a partnership with the Rio Potiribu, and I publicly apologize for not managing to incorporate this study here in time. However, an extraordinary flood crossed all our paths.

\par The largest flood ever recorded in Rio Grande do Sul, due to instabilities in the global climate system, significantly delayed my progress on this text. For this reason, I need to thank friends who stood by me in a traumatic scenario: Guilherme, \textit{O Kura}; Laura Azeredo; and José Augusto Müller Neto. I also thank all the professors, researchers, and journalists who, together, created a serious information network during the disaster.

\par I dedicate this thesis to the network of friendships that became the central pillar of my life during these graduate years. I am immensely grateful to friends for their support, company, and shared joys. Silly laughs, jokes, and hugs that brightened the dark days of the pandemic and continued to be the reason not to live in the mountains. I would like to name everyone who made me happier during this period, but that would be unfair -- I would forget someone. But it would also be unfair not to name certain special people. Thus, I thank Joana Winckler, Luiza Tonial, and Luisa Sarmento, especially for making IpoFest happen. I thank Lucia Torres, Thomas Silveira, and João Paulo Niedererauer for their pandemic company. I thank Clara Martinez, for everything, obviously. I thank Mariana Vivian, especially for the virtual journeys with Karl Popper, but not only. I thank Rafaela Machado, for guiding me through Berlin and Tapera. I thank Júlia Kuse, for the trails and trips that didn’t happen. I thank Ananda Casanova and little Alice. I thank my beloved friends Bettina Rubin, Matheus Schia, and Giovana Corsetti. I thank Mateus Coimbra and Santiago Costa, for our ship. I thank Luísa Acauan, Luciana Ruy, and Karina Kerne, for the mornings in the ocean.

\par With affection, I thank Fernanda Prestes for her company during my retreat from the world.

\par I thank my father, for giving me critical thinking, and my mother, for giving me creativity. Together, they also provided all the material and immaterial support to get here.

\par Finally, setting aside my Humanism, I thank Cantagalo, the fig trees, the howler monkeys, the noise of the stream, the vultures and hawks, the guinea pigs, and the snakes. I thank the Sun, the Moon, Venus, Mars, Jupiter, and Saturn, with whom I conversed for almost two years during the pandemic. I miss it a lot. I thank the Atlantic Ocean and its breath. At the Tramandaí River mouth, I thank the dolphins, especially Geraldona. Keep strong. At Zimba, I thank the right whales and turtles. I thank the Santa Marta Lighthouse and the Sambaquis and indigenous spirits, with whom I needed to make a pact. But mainly, I thank a certain wave that opened on the left, at Malvina, and made me see some things \textit{above and beyond}.


\clearpage

\end{document}