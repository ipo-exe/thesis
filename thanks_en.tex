\documentclass[./main_en.tex]{subfiles}
\graphicspath{{\subfix{./figs}}}

% ------------ main document ------------
\begin{document}
\doublespacing % line spacing
\large

% style setup
\newpage
\renewcommand{\headrulewidth}{0pt}
\thispagestyle{fancy}
%... then configure it
\fancyhf{} % Clear all header and footer fields.
\fancyfoot{} % clear all footer fields
\fancyfoot[C]{\thepage}

\begin{center}
    \sffamily{\Large \textbf{Acknowledgments}}
    \vspace{1mm}
\end{center}
\singlespacing
\normalsize

% custom paragraph skip
\setlength{\parskip}{\myparskip}

\par I am grateful to Professor and advisor Guilherme Fernandes Marques for the teachings, encouragement, partnership, and, above all, the trust placed in me since 2018. This has been a seven-year journey. We are already seeing some fruits ripen, but it is only the beginning.

\par I am grateful to my academic colleagues from \texttt{GESPLA}, the research group on water resources planning and management at the \texttt{IPH}. For the lunches and dinners at the University Restaurant, the jokes, and the company during focused work afternoons. Among the seasoned and new members I had the chance to work with, my thanks go especially, though not exclusively, to Ana Paula Dalcin, Ariane Sigallis, Giúlia Carrard, Gláucio Souza, Ivo Melo, Júlia Daiello, Juliano Finck, Julieta Nhampossa, Luísa Lucchese, Márcio Inada, Paola Kuele, Rossano Belladona and Vicente Lutz.

\par I also thank my graduate program colleagues at \texttt{IPH} who tolerated my frequent intrusions, asking what they were doing and how much they \textit{truly} believed in it. To those responsible for the Water Science Student Seminar, who consistently elected me as the go-to speaker for all editions. I extend my gratitude, in particular, to those who formed the Casa dos Brothers in Aracaju: Bruno Abatti, Carlos Ferrari, Cléber Gama, Juliana Andrade, Lara Nonnemacher, Marina Fagundes, and Priscila Kipper (this is, finally, the IpoTese). I am also deeply thankful to my friend Rafael Barbedo, \textit{The Veredas}, for his partnership and for sharing his knowledge about \texttt{HAND} during bohemian nights in Vienna, Berlin, and Bom Fim. To my brilliant friends from \texttt{USP} (who are not from \texttt{IPH}), I appreciate the friendship and invaluable exchanges in Vienna, Belo Horizonte, and Aracaju: Dimaghi Schwamback, Julien Sone, and Rodrigo Perdigão.

\par I am grateful to the network of mentors who guided me before graduate school. At age 19, in 2011, I began my scientific initiation, assisting the now-professor Maria Cristina de Almeida with bio-hydrogen experiments. At that time, the late professor Luiz Monteggia's guidance left a lasting impression of inventiveness and curiosity. During this period, I also interacted with Ayan Fleischmann, whom I thank for many fruitful exchanges, even during graduate school. I thank the late professor Dieter Wartchow for his inspiration in the pursuit of socially inclusive engineering. I am also grateful to professor Rualdo Menegat, who inspired me, especially with the Environmental Atlas of Porto Alegre, an encyclopedic and spectacularly illustrated work. Motivated by the Atlas, I try to invoke the spirit of Humboldt and Lovelock here. I thank professor Tatiana Silva, as well as colleagues from \texttt{LABMODEL} and \texttt{FURG}, for the practical experiences beyond university walls -- and for transforming me from a future microbiologist into \say{the map guy}. I am grateful to my last advisor, professor Fernando Dornelles, who brought me back to \texttt{IPH} at the end of my undergraduate studies in 2016.

\par I also feel indebted to the citizens of the Federative Republic of Brazil, whose taxes sustain the academic infrastructure that I was fortunate enough to benefit from and, in several projects, to be funded by. I am therefore thankful to the National Water and Basic Sanitation Agency, the Municipality of São Leopoldo, and the Rio Grande do Sul Sanitation Company for funding my scholarships during my doctorate. In particular, I am retroactively grateful to the administration of President Dilma Rousseff, who, in 2014, created the now-defunct Science Without Borders academic exchange program. Without the opportunity to study in Philadelphia, USA, the challenges I faced with the English language would have hindered my deep dive into the literature for this thesis. Even in global quarantine, without leaving home, I could explore knowledge thanks to the internet, the \texttt{CAPES} system, and proficiency in English.

\par I am grateful to Marcelo Kronbauer and the entire \texttt{UNISC} team for their friendship and for the opportunity to apply my studies in the Arroio Castelhano basin, even before finishing the thesis. We have much work ahead.

\par I thank the professors at \texttt{IPH} who accompanied me through the doctorate. Especially professor Walter Collischonn, for the enriching exchanges along this journey and for the opportunity to teach about \texttt{TOPMODEL} in the Hydrological Simulation course, even as a graduate student. I am also grateful to professor Rodrigo Paiva, for his insights and for helping me break protocol at the \texttt{SBRH} conference in Belo Horizonte by asking a live question to professor Keith Beven. I also thank professor Anderson Ruhoff, who showed me the path to real evapotranspiration estimates, and professor Nilza de Castro, for trusting me to create a case study with the Potiribu River. I apologize publicly for not managing to incorporate this study here in time, as an extraordinary flood crossed all our paths.

\par The largest recorded flood in Rio Grande do Sul, caused by global climate instabilities, substantially delayed my progress in this text. I am therefore grateful to friends who were present during this traumatic time: Guilherme, \textit{The Kura}; Laura Azeredo; and José Augusto Müller Neto. I am also grateful to all professors, researchers, and journalists who together formed the information network during the disaster.

\par I dedicate this thesis to the network of friendships that became the center of my life during these years of graduate studies. I am deeply grateful to my friends for their support, companionship, and shared joy. Silly laughter, jokes, and hugs that lit up the dark days of the pandemic and continued to be the reason not to move to the mountains. I would like to mention all those who made me happier during this period, but it would be unfair -- I would inevitably forget someone. Still, I especially thank Joana Winckler, Luiza Tonial, and Luisa Sarmento for making IpoFest happen. To Lucia Torres, Thomas Silveira, and João Paulo Niedererauer, for the pandemic companionship. To Clara Martinez, for everything, from Imbé to Barcelona. To Mariana Vivian, mainly for the virtual discussions with Karl Popper, but not only that. To Rafaela Machado, for guiding me through Berlin and Tapera. To Júlia Kuse, for the trails that happened and the trips that didn’t. To Ananda Casanova and little Alice, for many things. To the caring friends Bettina Rubin, Matheus Schia, and Giovana Corsetti. To Mateus Coimbra and Santiago Costa, for our spaceship. To Luísa Acauan, Luciana Ruy, and Karina Kerne, for the ocean mornings.

\par With affection, I thank dear Fernanda Prestes for her companionship during my retreat from the world.

\par I thank my father, Genuir, for giving me critical thinking, and my mother, Maria Lucia, for giving me creativity. Together, they also gave me all the material and immaterial support to get here.

\par Finally, admitting my Animist side, I thank Cantagalo, the fig trees, the howler monkeys, the aracuãs, the saracuras, the sound of the stream and the wind, the vultures and hawks, the preás and snakes. Loba, Bela, and Duque. I thank the Sun, the Moon, Venus, Mars, Jupiter, and Saturn, with whom I conversed for almost two years during quarantine. I miss that greatly. I thank the Atlantic Ocean and its breath. At the mouth of the Tramandaí River, I thank the dolphins and orcas, especially Geraldona. Stay strong. At Zimba, I thank the right whales and turtles, the rock that separates Luz from Barrinha, which is the center of the Universe. To the Diva's cliff, thank you very much. I thank the Santa Marta Lighthouse, the Sambaquis, and the indigenous spirits with whom I had to make an agreement. But most importantly, I thank a certain wave that opened to the left, at Malvina, and made me see some things \textit{above and beyond}.



\clearpage

\end{document}