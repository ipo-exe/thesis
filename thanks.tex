\documentclass[./main.tex]{subfiles}
\graphicspath{{\subfix{./figs}}}

% ------------ main document ------------
\begin{document}
\doublespacing % line spacing
\large

% style setup
\newpage
\renewcommand{\headrulewidth}{0pt}
\thispagestyle{fancy}
%... then configure it
\fancyhf{} % Clear all header and footer fields.
\fancyfoot{} % clear all footer fields
\fancyfoot[C]{\thepage}

\begin{center}
    \sffamily{\Large \textbf{Agradecimentos}}
    \vspace{5mm}
\end{center}
\singlespacing
\normalsize

% custom paragraph skip
\setlength{\parskip}{\myparskip}

\par Agradeço ao professor e orientador Guilherme Fernandes Marques pelos ensinamentos, incentivo, parceria e, principalmente, pela confiança depositada em mim desde 2018. Essa é uma trajetória de sete anos. Já estamos vendo alguns frutos ficarem maduros. Mas é o só o começo.

\par Agradeço aos colegas acadêmicos do \texttt{GESPLA}, grupo de pesquisa em planejamento e gestão de recursos hídricos no \texttt{IPH}. Pelos almoços e jantas no Restaurante Universitário, pelas brincadeiras, pela companhia nas tardes de trabalho. Entre a velha e nova guarda com quem convivi, meus agradecimentos vão especialmente para Ana Paula Dalcin, Ariane Sigallis, Gláucio Souza, Ivo Melo, Júlia Daiello, Giúlia Carrard, Juliano Finck, Julieta Nhampossa, Luísa Lucchese, Márcio Inada, Paola Kuele, Rossano Belladona e Vicente Lutz.

\par Agradeço aos colegas do programa de pós-graduação do \texttt{IPH} que me aturaram nas vezes que invadi as salas, perguntando sobre o que faziam e se acreditavam \textit{mesmo} naquilo. Aos organizadores do Seminário Discente da Ciência da Água, que me elegeram como palestrante coringa. Em especial, aos colegas que formaram a Casa dos Brothers em Aracaju: Bruno Abatti, Carlos Ferrari, Cléber Gama, Juliana Andrade, Lara Nonnemacher, Marina Fagundes e Priscila Kipper (aqui está, finalmente, a IpoTese). Agradeço também ao amigo Rafael Barbedo, \textit{O Veredas}, pela parceria e pelas discussões sobre o \texttt{HAND} em noites boêmias em Viena, Berlim e no Bom Fim. Aos amigos \say{da \texttt{USP}} (não do \texttt{IPH}), pela amizade e trocas que tivemos em Viena, Belo Horizonte e Aracaju: Dimaghi Schwamback, Julien Sone e Rodrigo Perdigão.

\par Agradeço à rede de mestres que me guiaram antes da pós-graduação. Aos 19 anos, em 2011, ingressei na iniciação científica, ajudando a hoje professora Maria Cristina de Almeida a preparar experimentos com bio-hidrogênio. A orientação do falecido professor Luiz Monteggia foi uma inspiração marcante de inventividade e curiosidade. Na época, também convivi com Ayan Fleischmann, que agradeço por trocas férteis já na pós-graduação. Agradeço ao falecido professor Dieter Wartchow, pela inspiração na busca por uma engenharia socialmente inclusiva. Agradeço ao professor Rualdo Menegat, que me inspirou com o Atlas Ambiental de Porto Alegre, uma obra enciclopédica ilustrada. Motivado pelo Atlas, tento conjurar o espírito de Humboldt e Lovelock. Agradeço à professora Tatiana Silva e aos colegas do \texttt{LABMODEL} e da \texttt{FURG}, pela prática além da Universidade -- e por terem me transformado de futuro microbiologista no \say{cara dos mapas}. Agradeço ao meu último orientador, o professor Fernando Dornelles, que me trouxe de volta ao \texttt{IPH} no final da graduação, em 2016.

\par Também agradeço aos cidadãos da República Federativa do Brasil, que sustentam a infraestrutura acadêmica que pude usufruir e, em vários projetos, ser financiado. Agradeço à Agência Nacional de Águas e Saneamento Básico, à Prefeitura de São Leopoldo e à Companhia Riograndense de Saneamento, pelas bolsas concedidas durante o doutorado. Agradeço retroativamente ao governo da Presidente Dilma Rousseff, que em 2014 criou o extinto programa Ciência Sem Fronteiras. Estudar em Filadélfia me permitiu superar dificuldades com o inglês, fundamentais para esta tese. Mesmo em quarentena, explorei o conhecimento graças à internet, ao sistema \texttt{CAPES} e ao domínio da língua inglesa.

\par Agradeço ao Marcelo Kronbauer e a toda equipe da \texttt{UNISC}, pela amizade e pela oportunidade de aplicar meus estudos na bacia do Arroio Castelhano, mesmo antes de concluir a tese. Temos muito trabalho pela frente.

\par Agradeço aos professores do \texttt{IPH} que me acompanharam no doutorado. Em especial, ao professor Walter Collischonn, pelas trocas férteis e pela oportunidade de lecionar sobre o \texttt{TOPMODEL} na disciplina de Simulação Hidrológica, mesmo sendo aluno da pós-graduação. Ao professor Rodrigo Paiva, pelas trocas e por me ajudar a quebrar o protocolo no \texttt{SBRH} de Belo Horizonte ao fazer uma pergunta ao vivo ao professor Keith Beven. Ao professor Anderson Ruhoff, que me ensinou a fazer estimativas de evapotranspiração real. À professora Nilza de Castro, pela confiança em criar um estudo de caso com o Rio Potiribu, e desculpo-me por não ter conseguido integrar esse estudo aqui a tempo, devido à enchente extraordinária que atravessou o caminho de todos nós.

\par A maior enchente já registrada no Rio Grande do Sul, decorrente das instabilidades no sistema climático global, atrasou meu progresso neste texto. Agradeço aos amigos que estiveram presentes em um cenário traumático: Guilherme, \textit{O Kura}; Laura Azeredo; e José Augusto Müller Neto. Agradeço também aos professores, pesquisadores e jornalistas que, juntos, formaram uma rede de informação durante o desastre.

\par Dedico esta tese à teia de amizades que se tornaram o centro da minha vida nesses anos de pós-graduação. Agradeço imensamente aos amigos pelo apoio, companhia e alegria compartilhada. Risadas, brincadeiras e abraços que iluminaram os dias sombrios da pandemia e me impediram de fugir para as montanhas. Gostaria de citar todos que me fizeram mais feliz nesse período, mas isso seria injusto -- eu acabaria esquecendo alguém. Contudo, agradeço especialmente Joana Winckler, Luiza Tonial e Luisa Sarmento, por fazerem a IpoFest acontecer. A Lucia Torres, Thomas Silveira e João Paulo Niedererauer, pela companhia pandêmica. Clara Martinez, por tudo, de Imbé a Barcelona. Mariana Vivian, principalmente pelas travessias virtuais com Karl Popper. Rafaela Machado, por me guiar por Berlim e Tapera. Júlia Kuse, pelas trilhas que aconteceram e as viagens que não. Ananda Casanova e a pequena Alice, por várias coisas. Aos amigos carijosos, Bettina Rubin, Matheus Schia e Giovana Corsetti. Mateus Coimbra e Santiago Costa, pela nossa nave. Luísa Acauan, Luciana Ruy e Karina Kerne, pelas manhãs no oceano.

\par Agradeço, com carinho, à querida Fernanda Prestes, pela companhia em meu retiro do mundo.

\par Agradeço ao meu pai, Genuir, por ter me dado o senso crítico, e à minha mãe, Maria Lucia, por ter me dado a criatividade. Juntos, eles também me deram todo o suporte material e imaterial para chegar aqui.

\par Por fim, admitindo meu lado Animista, agradeço ao Cantagalo, às figueiras, aos bugios, às aracuãs, às saracuras, ao barulho da sanga e do vento, aos urubus e gaviões, às preás e serpentes. Loba, Bela e Duque. Agradeço ao Sol, à Lua, a Vênus, a Marte, a Júpiter e a Saturno, que conversei por quase dois anos durante a quarenta. Sinto muita falta disso. Agradeço ao Oceano Atlântico e sua respiração. Na barra do Rio Tramandaí, agradeço aos botos e às orcas, principalmente a Geraldona. Sigam firmes. Na Zimba, agradeço às baleias francas e tartarugas, à pedra que separa o Luz da Barrinha, que é o centro do Universo. Ao costão da Diva, muito obrigado. Agradeço ao Farol da Santa Marta e aos Sambaquis e aos espíritos indígenas, que precisei fazer um acordo. Mas principalmente, agradeço a uma certa onda que veio abrindo pela esquerda, na Malvina, e me fez enxergar algumas coisas \textit{acima e além}. 

\clearpage

\end{document}