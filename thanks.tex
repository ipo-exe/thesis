\documentclass[./main.tex]{subfiles}
\graphicspath{{\subfix{./figs}}}

% ------------ main document ------------
\begin{document}
\doublespacing % line spacing
\large

% style setup
\newpage
\renewcommand{\headrulewidth}{0pt}
\thispagestyle{fancy}
%... then configure it
\fancyhf{} % Clear all header and footer fields.
\fancyfoot{} % clear all footer fields
\fancyfoot[C]{\thepage}

\begin{center}
    \sffamily{\Large \textbf{Acknowledgments}}
    \vspace{5mm}
\end{center}
\singlespacing
\normalsize

% custom paragraph skip
\setlength{\parskip}{\myparskip}

\par Agradeço ao professor e orientador Guilherme Fernandes Marques pelos ensinamentos, incentivo, parceria e, principalmente, pela confiança depositada em mim desde 2018. Essa é uma trajetória de sete anos. Já estamos vendo alguns frutos ficarem maduros. Mas é o só o começo.

\par Agradeço aos colegas acadêmicos do \texttt{GESPLA}, o grupo de pesquisa em planejamento e gestão de recursos hídricos no \texttt{IPH}. Pelos almoços e jantas no Restaurante Universitário, pelas brincadeiras e piadas, pela companhia em tardes concentradas de trabalho. Entre a velha e nova guarda que tive a oportunidade de conviver, meus agradecimentos vão principalmente, mas não somente, para Ana Paula Dalcin, Ariane Sigallis, Gláucio Souza, Júlia Daiello, Giúlia Carrard, Juliano Finck, Julieta Nhampossa, Luísa Lucchese, Márcio Inada, Paola Kuele, Rossano Belladona.

\par Agradeço aos meus colegas do programa de pós-graduação do \texttt{IPH} que me aturaram quando eu invadia as salas, perguntando o que eles faziam e o quanto eles acreditavam \textit{mesmo} naquilo que estavam fazendo. Aos responsáveis pelo Seminário Discente da Ciência da Água, que me elegeram o palestrante coringa de todas as edições. Com certo carinho especial, agradeço aos colegas que fizeram a Casa dos Brothers em Aracaju: Bruno Abatti, Carlos Ferrari, Cléber Gama, Juliana Andrade, Lara Nonnemacher, Marina Fagundes e Priscila Kipper (aqui está, finalmente, a IpoTese). Agradeço imensamente também o colega e amigo Rafael Barbedo, \textit{O Veredas}, pela parceria e por compartilhar os conhecimentos sobre o \texttt{HAND} nas noites boêmias de Viena, de Berlim e do Bom Fim. Ao colegas geniais da \say{da \texttt{USP}} (ou seja, que não são do \texttt{IPH}), agradeço pela amizade e excelentes trocas que tivemos em Viena, Belo Horizonte e Aracaju: Dimaghi Schwamback, Julien Sone e Rodrigo Perdigão.

\par Agradeço à rede de mestres que me guiaram antes da pós-graduação. Aos 19 anos, em 2011, ingressei na iniciação científica, quando ajudei a hoje professora Maria Cristina de Almeida a preparar experimentos com bio-hidrogênio. Nesse momento, a orientação do falecido professor Olinto Monteggia foi uma inspiração marcante de inventividade e curiosidade. Nessa época, também convivi com o Ayan Fleischmann, que agradeço também por trocas férteis já na pós-graduação. Agradeço ao falecido professor Dieter Wartchow, pela inspiração que tocou a mim na busca por uma engenharia socialmente inclusiva. Agradeço ao professor Rualdo Menegat, por me inspirar, principalmente, com o Atlas Ambiental de Porto Alegre, uma obra enciclopédica espetacularmente ilustrada. Motivado pelo Atlas, tento conjurar aqui o espírito de Humboldt e Lovelock. Agradeço à professora Tatiana Silva, e aos colegas do \texttt{LABMODEL} e da \texttt{FURG}, pela experiência prática além da Universidade -- e por terem de me transformado de futuro microbiologista no \say{cara dos mapas}. Agradeço ao meu último orientador, o professor Fernando Dornelles, que me trouxe de volta para o \texttt{IPH} no final da minha graduação, em 2016. 

\par Também sinto dever de agradecer aos cidadãos da República Federativa do Brasil, que sustentam por meio de impostos a infraestrutura acadêmica que tive a sorte de usufruir e, em diversos projetos, ser financiado. Assim, agradeço à Agência Nacional de Águas e Saneamento Básico, à Prefeitura de São Leopoldo e a Companhia Riograndense de Saneamento, por terem me concedido bolsas durante período do doutorado. Em especial, agradeço retroativamente ao governo da Presidente da República Dilma Rousseff, que em 2014 criou um programa de intercâmbio acadêmico, o Ciência Sem Fronteiras, hoje extinto. Sem a sorte de estudar em Filadélfia, nos Estados Unidos, as dificuldades que eu tinha com o inglês seriam proibitivas para mergulhar na literatura como eu fiz nesta tese. Mesmo em quarentena global, sem sair de casa, pude explorar o conhecimento graças à internet, ao sistema \texttt{CAPES} e ao domínio da língua inglesa.

\par Aos professores que me acompanharam de alguma forma no doutorado, ficam meus agradecimentos aos professores Walter Collischonn e Rodrigo Paiva, que foram da minha banca de qualificação, e ao professor Anderson Ruhoff, que me mostrou o caminho de como fazer estimativas de evapotranspiração real. Agradeço também à professora Nilza de Castro, pela confiança em criar uma parceria com o Rio Potiribu, e peço desculpas publicamente por não ter conseguido assimilar esse estudo aqui a tempo. A justificativa, porém, foi que uma enchente extraordinária atravessou o caminho de todos nós.

\par A maior enchente já registrada no Rio Grande do Sul, decorrente de instabilidades no sistema climático global, atrasou substancialmente meu progresso neste texto. Por isso preciso agradecer aos amigos que se fizeram presentes em um cenário traumático: o Guilherme, \textit{O Kura}; a Laura Azeredo; e o José Augusto Müller Neto. Também agradeço a todos os professores, pesquisadores e jornalistas que, juntos, criaram uma rede de informação séria durante o desastre.

\par Dedico essa tese à teia de amizades que se transformaram no pilar central da minha vida durante esses anos de pós-graduação. Agradeço imensamente aos amigos pelo apoio, companhia e alegrias compartilhadas. Risadas idiotas, brincadeiras e abraços que iluminaram os dias sombrios da pandemia e continuaram a ser a razão de não ir morar nas montanhas. Eu gostaria de citar todos que me fizeram mais alegre nesse período, mas isso seria injusto -- eu iria esquecer alguém. Mas também seria injusto não nominar certas pessoas especiais. Assim, agradeço à Joana Winckler, Luiza Tonial e Luisa Sarmento, em especial por fazerem a IpoFest acontecer. Agradeço à Lucia Torres, Thomas Silveira e o João Paulo Niedererauer, pela companhia pandêmica. Agradeço à Clara Martinez, por tudo, óbvio. Agradeço à Mariana Vivian, principalmente pelas travessias virtuais com Karl Popper, mas não somente. Agradeço à Rafaela Machado, por me guiar por Berlim e pela Tapera. Agradeço à Júlia Kuse, pelas trilhas e viagens que não aconteceram. Agradeço à Ananda Casanova e a pequena Alice. Agradeço aos amigos carijosos, Bettina Rubin, Matheus Schia e Giovana Corsetti. Agradeço ao Mateus Coimbra e ao Santiago Costa, pela nossa nave. Agradeço à Luísa Acauan, Luciana Ruy e Karina Kerne, pelas manhãs no oceano.

\par Agradeço, com carinho, à Fernanda Prestes, pela companhia em meu retiro do mundo.

\par Agradeço ao meu pai, por ter me dado o senso crítico, e à minha mãe, por ter me dado a criatividade. Juntos, eles também me deram todo o suporte material e imaterial para chegar aqui.

\par Por fim, deixando de lado o meu Humanismo, agradeço ao Cantagalo, às figueiras, aos bugios, ao barulho da sanga, aos urubus e gaviões, às preás e serpentes. Agradeço ao Sol, à Lua, a Vênus, a Marte, a Júpiter e a Saturno, que conversei por quase dois anos durante a pandemia. Sinto muita falta disso. Agradeço ao Oceano Atlântico e sua respiração. Na barra do Rio Tramandaí, agradeço aos botos, principalmente a Geraldona. Sigam firmes. Na Zimba, agradeço às baleias francas e tartarugas. Agradeço ao Farol da Santa Marta e aos Sambaquis e aos espíritos indígenas, que precisei fazer um acordo. Mas principalmente, agradeço a uma certa onda que veio abrindo pela esquerda, na Malvina, e me fez enxergar algumas coisas \textit{acima e além}. 

\clearpage

\end{document}