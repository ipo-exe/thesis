\documentclass[./main.tex]{subfiles}
\graphicspath{{\subfix{./figs}}}

% ------------ main document ------------
\begin{document}
\large

% style setup
\newpage
\renewcommand{\headrulewidth}{0pt}
\thispagestyle{fancy}
%... then configure it
\fancyhf{} % Clear all header and footer fields.
\fancyfoot{} % clear all footer fields
\fancyfoot[C]{\thepage}

\begin{center}
    \sffamily{\Large \textbf{Resumo}}
    \vspace{5mm}
\end{center}
\singlespacing
\normalsize
Esta tese é uma síntese que organiza as bases conceituais e filosóficas para o uso de modelos hidrológicos na conservação de bacias hidrográficas e na expansão da infraestrutura verde. Com o objetivo de criar um mapa de ideias integrado, o trabalho não apenas conecta fundamentos teóricos e práticos, mas oferece um guia estruturado para futuras investigações e aplicações na área. O capítulo inicial apresenta as bases epistemológicas, abordando as justificações e limitações dos modelos hidrológicos. O segundo capítulo desenvolve uma abordagem ontológica com a Dinâmica de Sistemas, explorando como a arquitetura dos modelos define as respostas hidrológicas e identificando pontos de intervenção estratégica. No terceiro capítulo, a tese revisa a evolução dos paradigmas hidrológicos, culminando na Teoria da Conectividade, que propõe uma unificação dos processos de escoamento superficial e subterrâneo. O último capítulo explora o papel da Economia Ecológica na gestão de bacias, utilizando o modelo \texttt{PLANS} em esquemas de Pagamentos por Serviços Ambientais (PSA) para priorizar áreas de conservação. Ao sintetizar esses conceitos, a tese estabelece um conhecimento estruturado que facilita o avanço de novas pesquisas, permitindo que a comunidade científica articule com maior clareza e objetividade soluções sustentáveis e adaptativas para a gestão de bacias hidrográficas.\\[2ex]
	
\noindent \textit{\textbf{palavras-chave}} --- Modelagem Hidrológica; Revitalização de Bacias Hidrográficas; Pagamentos por Serviços Ambientais.
\clearpage

\large
\begin{center}
    \sffamily{\Large \textbf{Abstract}}
    \vspace{5mm}
\end{center}
\singlespacing
\normalsize
This thesis is a synthesis that organizes the conceptual and philosophical foundations for the use of hydrological models in watershed conservation and the expansion of green infrastructure. With the aim of creating an integrated map of ideas, the work not only connects theoretical and practical foundations but also offers a structured guide for future investigations and applications in the field. The initial chapter presents the epistemological bases, addressing the justifications and limitations of hydrological models. The second chapter develops an ontological approach with Systems Dynamics, exploring how model architecture defines hydrological responses and identifying points for strategic intervention. In the third chapter, the thesis reviews the evolution of hydrological paradigms, culminating in the Theory of Connectivity, which proposes an unification of surface and subsurface flow processes. The final chapter explores the role of Ecological Economics in watershed management, using the \texttt{PLANS} model in Payment for Ecosystem Services (PES) schemes to prioritize conservation areas. By synthesizing these concepts, the thesis establishes a structured body of knowledge that facilitates the advancement of new research, enabling the scientific community to articulate sustainable and adaptive solutions for watershed management with greater clarity and objectivity.\\[2ex]
	
\noindent \textit{\textbf{keywords}} --- Hydrological Modeling; Watershed Conservation; Payments for Ecosystem Services.

\clearpage
\end{document}