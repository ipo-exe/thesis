\documentclass[./main.tex]{subfiles}
\graphicspath{{\subfix{./figs}}}

% ------------ main document ------------
\begin{document}
\doublespacing % line spacing
\large

% style setup
\newpage
\renewcommand{\headrulewidth}{0pt}
\thispagestyle{fancy}
%... then configure it
\fancyhf{} % Clear all header and footer fields.
\fancyfoot{} % clear all footer fields
\fancyfoot[C]{\thepage}

\begin{center}
    \sffamily{\Large \textbf{Nota sobre os tempos verbais}}
    \vspace{5mm}
\end{center}
\singlespacing
\normalsize

\noindent Ao revisar o texto, eu percebi uma dificuldade da minha parte em manter a consistência com os tempos verbais. Com o espírito de amenizar a leitura (e literalmente escrever de forma correta), adotei uma convenção, descrita a seguir.

\vspace{10mm}

\noindent Esta tese é uma \textbf{síntese} sobre algumas ideias. Por simplesmente existirem, as ideias \textit{em si} são conjugadas no \sethlcolor{yellow}\hl{presente}:

\begin{adjustwidth}{100pt}{0pt}
\medskip
\small
A inferência dedutiva \sethlcolor{yellow}\hl{garante} a verdade da sentença consequente desde que suas premissas antecedentes sejam verdadeiras
\medskip
\end{adjustwidth}

\noindent A \textbf{narrativa} de como essas ideias foram concebidas, por fazer referência a uma sequência de proposições ao longo da História, é conjugada no \sethlcolor{lime}\hl{passado}:

\begin{adjustwidth}{100pt}{0pt}
\medskip
\small
Popper \sethlcolor{lime}\hl{estava} ciente da gravidade do problema da indução (...)
\medskip
\end{adjustwidth}

\noindent Porém, quando eu busco apreciar o \textbf{conteúdo} de uma publicação (um livro, um artigo, etc.), mesmo que publicada faz séculos, eu volto a falar de ideias que simplesmente existem, no \sethlcolor{yellow}\hl{presente} (afinal, as ideias ainda estão ali, basta ler a publicação):

\begin{adjustwidth}{100pt}{0pt}
\medskip
\small
(...) o que Popper \sethlcolor{yellow}\hl{demonstra} é que existe uma assimetria fundamental entre esses dois modos de lógica (...)
\medskip
\end{adjustwidth}

\noindent Mas alguns autores descrevem \textbf{experiências empíricas}, eventos singulares que aconteceram. Assim, as experiências empíricas \textit{em si} são conjugadas no \sethlcolor{lime}\hl{passado}:

\begin{adjustwidth}{100pt}{0pt}
\medskip
\small
Nesse mesmo estudo, os autores \sethlcolor{yellow}\hl{relatam} que o escoamento subsuperficial \sethlcolor{lime}\hl{foi} responsável por cerca de 97\% da resposta rápida nas enchentes.
\medskip
\end{adjustwidth}

\vspace{10mm}

\noindent Se a convenção acima não foi completamente seguida, a causa foi a minha incompetência.




\clearpage

\end{document}