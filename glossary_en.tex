\makeglossaries

% pragmatic realism
\newglossaryentry{prag-realism}
{
    name=pragmatic realism,
    description={Term proposed by Keith Beven to describe the implicit realism commonly held by environmental model users. In this philosophy, it is accepted that models provide approximate representations of reality and can improve as new technologies become available}
}

% rationalism
\newglossaryentry{rationalism}
{
    name=rationalism,
    description={Philosophical doctrine that supports the superiority of deductive, intuitive, and innate logic to human knowledge, justifying the truth of theories. This doctrine opposes empiricism}
}

% empiricism
\newglossaryentry{empiricism}
{
    name=empiricism,
    description={Philosophical doctrine that argues that all knowledge originates from empirical experience, that is, observations of the external world. This doctrine opposes rationalism}
}

% problem of justification
\newglossaryentry{problem-just}
{
    name=problem of justification,
    description={The challenge of establishing the truth of a particular piece of knowledge or theory}
}

% theory
\newglossaryentry{teoria}
{
    name=theory,
    description={A universal statement (or system of statements) that definitively establishes the truth of a phenomenon}
}

% hypothesis
\newglossaryentry{hipotese}
{
    name=hypothesis,
    description={A universal statement in a trial phase, aiming to be elevated to the status of theory after confirmation or corroboration}
}

% deductive inference
\newglossaryentry{infer-dedu}
{
    name=deductive inference,
    description={Logical reasoning that establishes the truth of a given statement from antecedent premises. The truth of the consequent statement is guaranteed only if the antecedent premises are also true}
}

% inductive inference
\newglossaryentry{infer-indu}
{
    name=inductive inference,
    description={Empirical reasoning based on generalization or extrapolation, establishing a universal statement from observations of singular statements. The truth of the universal statement is not guaranteed but presents degrees of probability}
}

% infinite regress problem
\newglossaryentry{problem-regress}
{
    name=infinite regress problem,
    description={The challenge of establishing the ultimate origin of logical or rational knowledge, given that all premises must be deduced from more fundamental premises, leading to an infinite (or circular) chain of premises}
}

% induction problem
\newglossaryentry{problem-indu}
{
    name=induction problem,
    description={Also known as \textbf{Hume's induction problem}. A circular invalid argument that arises from the justification of inductive knowledge through the principle of uniformity, as it invokes inductive knowledge to support itself}
}

% principle of uniformity
\newglossaryentry{principio-uniform}
{
    name=principle of uniformity,
    description={Assumption that the same natural regularities observed empirically in the past will be the same in the future, i.e., that nature is predictable based on its past and that no arbitrary changes will occur in its laws (for example, the Earth suddenly stopping its rotation)}
}

% degree of conviction
\newglossaryentry{grau-convic}
{
    name=credence,
    description={Measure of belief. Central concept in Bayesian epistemology derived from the idea that knowledge is not a matter of all-or-nothing, but exhibits subtleties between true and false. This concept can be considered a probability under certain circumstances}
}

% space of possibilities
\newglossaryentry{espaco-possib}
{
    name=space of possibilities,
    description={The set of possibilities generated between hypotheses and evidence in Bayesian epistemology. For probability mathematics to apply to this set, the possibilities must be \textit{mutually exclusive} (cannot be true at the same time) and \textit{collectively exhaustive} (at least one is true)}
}

% principle of probabilism
\newglossaryentry{princip-prob}
{
    name=principle of probabilism,
    description={Principle used in Bayesian epistemology to treat degrees of conviction as probabilities. It has three axioms: non-negativity; normalization; and additivity}
}

% prior probability
\newglossaryentry{prior}
{
    name=prior,
    description={The probability that the hypothesis $H$ is true before considering the probability that the favorable evidence $E$ is true. Denoted as $P(H)$}
}

% posterior probability
\newglossaryentry{posterior}
{
    name=posterior,
    description={The probability that the hypothesis $H$ is true after considering the probability that the favorable evidence $E$ is true. Denoted as $P(H | E)$}
}

% likelihood
\newglossaryentry{likelihood}
{
    name=likelihood,
    description={The probability that the evidence $E$ is true after considering the probability that the hypothesis $H$ is true. Denoted as $P(E | H)$}
}

% principle of conditionalization
\newglossaryentry{princip-cond}
{
    name=principle of conditionalization,
    description={Principle used in Bayesian epistemology to update degrees of conviction in hypotheses based on evidence. To maintain consistency with the principle of probabilism, conditionalization involves zeroing, scaling, and normalizing the values of updated probabilities}
}

% Bayes' theorem
\newglossaryentry{bayes-theorem}
{
    name=Bayes' theorem,
    description={Mathematical formulation for determining the posterior probability: $P(H | E) = P(H) \cdot P(E | H) / P(E)$, meaning $\text{Posterior} = \text{Prior} \times \text{Likelihood} \div \text{Evidence}$}
}

% conditioning
\newglossaryentry{conditioning}
{
    name=conditioning,
    description={[synonym of conditionalization] Application of Bayes' Theorem to a hypothesis to update its degree of conviction. It can be done in successive steps as new evidence is obtained}
}

% logical positivism
\newglossaryentry{positivismo}
{
    name=logical positivism,
    description={An empiricist philosophical movement from the early 20th century, also known as logical empiricism}
}

% problem of priors
\newglossaryentry{problem-priors}
{
    name=problem of priors,
    description={The difficulty of justifying the initial definition of degrees of conviction in a hypothesis before any evidence is obtained. In Bayesian epistemology, solutions to this problem are proposed mainly through two approaches: objective and subjective}
}

% subjective Bayesianism
\newglossaryentry{bayes-sub}
{
    name=subjective Bayesianism,
    description={An approach in Bayesian epistemology that argues that any prior distribution is valid as long as it does not violate the principle of probabilism}
}

% objective Bayesianism
\newglossaryentry{bayes-obj}
{
    name=objective Bayesianism,
    description={An approach in Bayesian epistemology that argues that the prior distribution must be defined in such a way as to observe the principle of indifference}
}

% principle of indifference
\newglossaryentry{princip-indif}
{
    name=principle of indifference,
    description={A principle adopted by the objective Bayesianism approach, stating that the degree of conviction in two or more hypotheses should be equal unless there are reasons to the contrary. In the case of complete ignorance, the prior distribution must be uniform}
}

% statistical uncertainty
\newglossaryentry{uncert-stats}
{
    name=statistical uncertainty,
    description={Uncertainty arising from random noise in the observed data. This type of uncertainty has stationary statistical characteristics that may or may not be structured with bias, heteroscedasticity, and autocorrelation. One way or another, this uncertainty can be modeled by probability distributions}
}

% rating curve
\newglossaryentry{rating-curve}
{
    name=rating curve,
    description={Functional relationship between the level and flow of a river or channel at a specific section. Typically, the following power function is used: $Q = a(h - h_0)^b$, where $Q$ is the flow; $h$ is the level; and $a$, $b$, and $h_0$ are parameters adjusted by observed data}
}



% model
\newglossaryentry{model}
{
    name=model,
    description={A model is a simplified representation of a real-world phenomenon, often used to explain, predict, or simulate various processes}
}

% central limit theorem
\newglossaryentry{theorem-central-limit}
{
    name=central limit theorem,
    description={A theorem that establishes the mathematical fact that the sample mean of any population exhibits a normal distribution. Regardless of the population’s distribution (uniform, normal, etc.), the mean obtained from samples will be normally distributed. This happens because the mean is a sum, and in sums of random numbers, low sampled values tend to offset high sampled values, resulting in a bell-shaped pattern similar to a normal distribution}
}

% Monte Carlo simulations
\newglossaryentry{monte-carlo}
{
    name=Monte Carlo simulations,
    description={A numerical method in which numerous statistically equivalent resamplings are performed to estimate the final behavior of a model involving random variables (i.e., when $n \to \infty$). The name Monte Carlo refers to a casino in Monaco, alluding to the idea of making numerous "rolls" to perform a robust statistical analysis}
}

% statistical model
\newglossaryentry{model-stats}
{
    name=statistical model,
    description={A statistical model is a specific theory about the mathematical behavior of data, without theoretical links to underlying phenomena}
}

% critical rationalism
\newglossaryentry{critical-rationalism}
{
    name=critical rationalism,
    description={A rationalist philosophical approach proposed by Karl Popper, which establishes falsifiability as the criterion for demarcating scientific theories. In this view, the power of empirical evidence lies in justifying, through deductive logic, the falsity of theories (never their truth). For instance, it takes just one black swan to disprove the theory that all swans are white. While unrefuted, theories are merely corroborated by evidence}
}

% demarcation problem
\newglossaryentry{problem-demarc}
{
    name=demarcation problem,
    description={The difficulty of establishing the difference between a scientific theory and a merely metaphysical theory, which relies solely on pure abstractions}
}

% falsifiability
\newglossaryentry{falseabilidade}
{
    name=falsifiability,
    description={The ability of a theory to be shown as false through empirical experience (observations and experiments). A falsifiable theory is not necessarily false but \textit{can} be proven false by empirical evidence. In critical rationalism, this ability is the criterion for determining whether a theory is scientific}
}

% context of justification
\newglossaryentry{contexto-justificacao}
{
    name=context of justification,
    description={A perspective in the philosophy of science that deals with the problem of justifying the truth of theories}
}

% context of discovery
\newglossaryentry{contexto-descoberta}
{
    name=context of discovery,
    description={A perspective in the philosophy of science that deals with the problem of understanding the historical change of theories}
}

% scientific community
\newglossaryentry{comunidade-cientifica}
{
    name=scientific community,
    description={The people who practice science at a given period in history. It can refer to the entirety of scientists or a specific subset within a field. Thomas Kuhn argues that, during certain historical periods, the scientific community is characterized by sharing a paradigm}
}

% paradigm
\newglossaryentry{paradigma}
{
    name=paradigm,
    description={A concept articulated by Thomas Kuhn referring to the set of exemplary solutions to research problems, i.e., a system of theories, instruments, and auxiliary practices that solve certain widely accepted problems and are promising for resolving open controversial problems with great competitive appeal}
}

% normal science
\newglossaryentry{ciencia-normal}
{
    name=normal science,
    description={A concept articulated by Thomas Kuhn referring to the historical period in which a given scientific community shares the same paradigm. Normal science tends to end in a crisis, followed by a revolution imposed by the advent of a new paradigm}
}

% theoretical incommensurability
\newglossaryentry{incomensu-theory}
{
    name=theoretical incommensurability,
    description={A concept articulated by Thomas Kuhn, referring to the problem of intellectual communication between theories under different paradigms. Two paradigms are fundamentally different, making comparison between their concepts precarious (even if they use the same name and mathematical symbol)}
}

% scientific realism
\newglossaryentry{realism-sci}
{
    name=scientific realism,
    description={A current in the philosophy of science that defends the thesis that the purpose of science is to provide theories that are true descriptions of reality}
}

% realism
\newglossaryentry{realism}
{
    name=realism,
    description={A metaphysical conception that admits the existence of objective reality, i.e., reality does not depend on anyone to observe it}
}

% idealism
\newglossaryentry{idealism}
{
    name=idealism,
    description={A metaphysical conception that opposes realism. In this perspective, which can have ontological or epistemological interpretations, reality is understood as a subjective product of the mind}
}

% instrumentalism
\newglossaryentry{instrument}
{
    name=instrumentalism,
    description={An empiricist radical philosophy of science that opposes scientific realism. This doctrine holds that the goal of science is to produce empirically adequate theories and nothing more. It argues that empirical adequacy does not imply a true description of reality}
}

% underdetermination problem
\newglossaryentry{problem-subdet}
{
    name=underdetermination problem,
    description={The difficulty of ensuring that the observed evidence determines the truth of a theory without there being other empirically equivalent theories}
}

% inference to the best explanation
\newglossaryentry{ibe}
{
    name=inference to the best explanation,
    description={[synonym of abduction] Non-deductive reasoning that seeks to define the hypothesis that best explains empirical evidence}
}

% heuristic
\newglossaryentry{heuristic}
{
    name=heuristic,
    description={A set of problem-solving techniques that do not guarantee an optimal or rational solution but are sufficient to achieve practical decision-making purposes. The main example is solving problems through trial and error}
}

% equifinality problem
\newglossaryentry{problem-equifinal}
{
    name=equifinality problem,
    description={A term coined by Keith Beven for the mild version of the underdetermination problem in the case of environmental numerical models. The underdetermination of models occurs because the information about the modeled processes is incomplete, ensuring the existence of model structures that are empirically equivalent, or equifinal}
}

% auxiliary hypotheses
\newglossaryentry{aux-hyp}
{
    name=auxiliary hypotheses,
    description={A set of hypotheses necessary in addition to the main hypothesis of a model}
}

% total error equation
\newglossaryentry{eq-total-error}
{
    name=total error equation,
    description={An equation that includes all sources of errors in a model, both statistical and epistemic}
}

% empirically equivalent models
\newglossaryentry{emp-eq-model}
{
    name=empirically equivalent models,
    description={Different models that yield simulated results with no significant deviations given the total observational error. In this case, there is no empirical reason to favor one model over another, at least concerning the main hypothesis of the models}
}

% overfitting problem
\newglossaryentry{problem-overfitting}
{
    name=overfitting problem,
    description={A problem that emerges in model calibration when a model is excessively fitted to the available empirical information, resulting in poorer performance when new empirical observations are evaluated}
}

% empirically acceptable model
\newglossaryentry{emp-acc-model}
{
    name=empirically acceptable model,
    description={A model that yields simulated results that satisfy empirical observations with a pre-established level of confidence}
}

% bracketing inequality
\newglossaryentry{eq-bracketing}
{
    name=bracketing inequality,
    description={Inequality used to test whether a model is empirically acceptable. The model’s simulated results must be bracketed by the observational uncertainty bands (total observational error) with a predefined level of confidence (rejection criterion)}
}


% calibration process
\newglossaryentry{proc-calib}
{
    name=calibration process,
    description={A procedure to adjust model parameters to increase its degree of confirmation against empirical evidence}
}

% empirical uncertainty
\newglossaryentry{uncert-empirical}
{
    name=empirical uncertainty,
    description={The scientific component of uncertainties in the decision-making process based on evidence. It consists of both epistemic and statistical uncertainties about the state of the world. Other non-empirical components include ethical and political uncertainty}
}

% evidence-based policies
\newglossaryentry{ebp}
{
    name=evidence-based policies,
    description={A concept in public policies that seeks total or partial support from objective evidence to guide decision-making and resource allocation}
}

% epistemic uncertainty
\newglossaryentry{uncert-episteme}
{
    name=epistemic uncertainty,
    description={A general concept referring to the various non-statistical uncertainties present in the modeling process. Unlike statistical uncertainty, which refers to the available information, epistemic uncertainty is associated with unavailable information}
}

% measurement error
\newglossaryentry{error-measure}
{
    name=measurement error,
    description={Statistical uncertainty resulting from the measurement of empirical evidence}
}

% commensurability error
\newglossaryentry{error-commensu}
{
    name=commensurability error,
    description={Epistemic uncertainty resulting from the difference in scales of time and space between observed processes and modeled processes}
}

% input data error
\newglossaryentry{error-input}
{
    name=input data error,
    description={Statistical and epistemic uncertainty associated with the data used to configure the model. For instance, rainfall data present statistical measurement uncertainty and the epistemic uncertainty of spatial interpolation}
}

% model structural error
\newglossaryentry{error-struct}
{
    name=model structural error,
    description={Epistemic uncertainty associated with the theoretical concepts and computational procedures employed in a given model}
}

% effective observational error
\newglossaryentry{error-obs}
{
    name=effective observational error,
    description={The overlap of measurement error and commensurability error, representing the uncertainty that a model must be subjected to in order to assess its empirical adequacy}
}

% target system
\newglossaryentry{sys-target}
{
    name=target system,
    description={A real system that a model supposedly seeks to represent, conveying a theory or hypothesis about this system}
}

% representation problem
\newglossaryentry{problem-repr}
{
    name=representation problem,
    description={The difficulty of constructing a model that performs the semantic or syntactic function of representing a target system}
}

% idealization
\newglossaryentry{idealization}
{
    name=idealization,
    description={A fundamental procedure used to construct models, making the representations more tangible and understandable than the target system itself}
}

% Aristotelian idealization
\newglossaryentry{idealiz-arist}
{
    name=Aristotelian idealization,
    description={(See abstraction). A method of idealization that uses abstraction, a process aimed at removing supposedly irrelevant factors and aspects of the target system, leaving only its essence}
}

% Galilean idealization
\newglossaryentry{idealiz-galil}
{
    name=Galilean idealization,
    description={A method of idealization that applies controlled distortions that could be incrementally removed to asymptotically reach the final behavior of the target system}
}

% abstraction
\newglossaryentry{abstraction}
{
    name=abstraction,
    description={An idealization process that seeks to remove supposedly irrelevant factors and aspects of the target system, leaving only its essence}
}

% negligibility premises
\newglossaryentry{neglig-premis}
{
    name=negligibility premises,
    description={A concept introduced by Musgrave (1980), referring to the process of ignoring known important causal factors during abstraction, i.e., when abstraction ends up presenting a model with known falsehoods}
}

% scale models
\newglossaryentry{scale-models}
{
    name=scale models,
    description={Representations that are literally copies of the target system at a scale suitable for human manipulation, whether scaled-down or enlarged}
}

% scale similarity
\newglossaryentry{scale-similarity}
{
    name=scale similarity,
    description={The ability to convert between the real scale of a target system and the scale of a reduced or enlarged model. Similarity is usually not complete, being valid only in certain aspects (e.g., geometrically similar, but not in terms of density or strength)}
}

% analog models
\newglossaryentry{analog-models}
{
    name=analog models,
    description={Representations based on an analogy with the target system, preferably involving systems with supposedly the same mathematical structure (formal analogy)}
}

% analogy
\newglossaryentry{analogy}
{
    name=analogy,
    description={A comparison between two or more objects, emphasizing their supposedly similar aspects. Analogy is used in modeling to idealize systems in terms of other, more tangible systems that supposedly have the same mathematical structure}
}

% analogical inference
\newglossaryentry{infer-analog}
{
    name=analogical inference,
    description={Non-deductive and non-inductive reasoning that concludes that a given object $O_1$ has the property $P_1$ of object $O_2$ because they share other properties}
}

% exploratory models
\newglossaryentry{explore-models}
{
    name=exploratory models,
    description={Models used in scientific research as tools to investigate and develop new hypotheses, especially useful in areas where established theories are insufficient or non-existent, allowing the exploration of theoretical possibilities and potential explanations}
}

% minimalist models
\newglossaryentry{mini-models}
{
    name=minimalist models,
    description={Models that are simplified to the extreme, used to understand complex phenomena by reducing them to the essentials, focusing on fundamental aspects without the complication of excessive details}
}

% system
\newglossaryentry{system}
{
    name=system,
    description={An emergent ontological entity defined by a set of fundamental parts that exhibit relationships with each other}
}

% hylomorphism
\newglossaryentry{hilomorphism}
{
    name=hylomorphism,
    description={A holistic ontological theory proposed by Aristotle, which states that all things are composed of both matter and form}
}

% feedback
\newglossaryentry{feedback}
{
    name=feedback,
    description={A recursive information loop that acts on a system. It can be positive, reinforcing a given process, or negative, stabilizing a given process}
}

% structural isomorphism
\newglossaryentry{struc-iso}
{
    name=structural isomorphism,
    description={A concept articulated by Ludwig von Bertalanffy to support General Systems Theory, being a formal analogy (homology) observed in different phenomena}
}

% open systems
\newglossaryentry{sys-open}
{
    name=open systems,
    description={Systems capable of processing an inflow and outflow of matter, energy, and information, in contrast to the closed systems described by classical thermodynamics}
}

% attractors
\newglossaryentry{atractors}
{
    name=attractors,
    description={A set of final, stable, or unstable behaviors observed in the solution of a given system of differential equations, depending on parameter values and initial conditions}
}

% mental models
\newglossaryentry{mental-models}
{
    name=mental models,
    description={A term from systems dynamics for subjective and personal models that are still in the early stages of the modeling process}
}

% deterministic chaos
\newglossaryentry{chaos}
{
    name=deterministic chaos,
    description={Extreme sensitivity produced by nonlinearities in dynamic systems, generally associated with initial conditions. A chaotic system evolves in a highly unstable manner, oscillating between various final states. Rounding errors can amplify this effect even more, though the origin of the process lies in the mathematical formulation itself}
}

% principle of computational irreducibility
\newglossaryentry{problem-irred}
{
    name=principle of computational irreducibility,
    description={The principle that, for many complex systems, there is no shortcut or simplified method to predict their future behavior faster than the step-by-step execution of the system itself}
}

% rounding error
\newglossaryentry{round-error}
{
    name=rounding error,
    description={The difference between the exact numerical value and the approximate value resulting from rounding, caused by limitations in precision in computational number representation}
}

% truncation error
\newglossaryentry{integration-error}
{
    name=truncation error,
    description={The difference between the exact value of a function or analytical mathematical calculation and its approximation resulting from the numerical method employed to calculate the value in a computational environment}
}

% strange attractor
\newglossaryentry{strange-atrc}
{
    name=strange attractor,
    description={A set of states in a dynamic system that, despite being chaotic, possesses a defined geometric structure and attracts the system’s trajectories, characterizing an ordered behavior within chaos}
}

% agent-based models
\newglossaryentry{abm-models}
{
    name=agent-based models,
    description={Agent-based models are computational simulation systems that use autonomous entities, with individual behaviors and interactions,}
}

% systems dynamics
\newglossaryentry{sys-dyn}
{
    name=Systems Dynamics,
    description={Systems dynamics is a modeling and analysis approach that uses feedback loops, levels, flows, and delays to understand the behavior of complex systems over time, helping to identify and predict behavior patterns and their underlying causes}
}
% compartment model
\newglossaryentry{compart-models}
{
    name=compartment model,
    description={In systems dynamics, a compartment model is a modeling technique that divides a system into different sectors, where each compartment represents an accumulated quantity (level) of a specific variable, and the rates of flow between these compartments describe the changes over time}
}

% causal structure
\newglossaryentry{causal-struct}
{
    name=causal structure,
    description={In systems dynamics, the causal structure refers to the set of cause-and-effect relationships that determine a system's behavior over time, including feedback loops and flows between the system's compartments}
}

% causal loop diagram
\newglossaryentry{causal-diag}
{
    name=causal loop diagram,
    description={In systems dynamics, a causal loop diagram is a visual tool that represents cause-and-effect relationships between the variables of a system, highlighting how changes in one variable influence others through reinforcing and balancing loops}
}

% system boundary
\newglossaryentry{bounds}
{
    name=system boundary,
    description={In systems dynamics, the system boundary defines the limits of what is included or excluded in a system analysis, specifying which compartments exert relevant causal effects on the system without being considered external factors}
}

% balance equation
\newglossaryentry{eq-balance}
{
    name=balance equation,
    description={A differential equation that establishes that the variation in a level results from the net effect of the inflow and outflow rates}
}

% conservation principle
\newglossaryentry{princip-conserv}
{
    name=conservation principle,
    description={A principle used in systems dynamics to apply balance equations to compartments, often referring to the conservation of mass or energy (in physical systems)}
}

% exogenous variables
\newglossaryentry{exo-vars}
{
    name=exogenous variables,
    description={In systems dynamics, exogenous variables are factors external to the modeled system that influence its behavior but are not affected by the system's internal dynamics. They are imposed from outside (external forces) and remain constant or follow a predetermined pattern during the simulation}
}

% parameters
\newglossaryentry{parameters}
{
    name=parameters,
    description={Fixed values of coefficients that define the characteristics and behaviors of elements and processes within a model. They are used to adjust the relationships and functions of the system, determining the system's response and dynamics under different conditions}
}

% numerical integration problem
\newglossaryentry{problem-numerics}
{
    name=numerical integration problem,
    description={The difficulty of obtaining exact values when solving balance equations in the simulation of dynamic systems on digital computers. See truncation error}
}

% Euler method
\newglossaryentry{method-euler}
{
    name=Euler method,
    description={A simple numerical integration technique used to solve ordinary differential equations, where the solution is approximated by advancing in small steps, using the known derivative to estimate the value of the function at the next point from the current value}
}

% temporal insensitivity principle
\newglossaryentry{princip-ins}
{
    name=temporal insensitivity principle,
    description={A guideline by John Sterman, within systems dynamics, that the results of model simulations should not be sensitive to the time step used in numerical integration, regardless of the method adopted}
}

% auxiliary equations
\newglossaryentry{eq-aux}
{
    name=auxiliary equations,
    description={Equations used in the programming of dynamic systems (procedural model) to break down into steps that are easier for humans to understand}
}

% supplementary equations
\newglossaryentry{eq-sup}
{
    name=supplementary equations,
    description={Equations used in the programming of dynamic systems (procedural model) to capture important information that is not part of the modeled system itself, such as statistics of variables}
}

% perceptual model
\newglossaryentry{percept-model}
{
    name=perceptual model,
    description={Also called a mental model, it consists of the subjective and highly personal representation of an individual about the target system (object)}
}

% conceptual model
\newglossaryentry{concept-model}
{
    name=conceptual model,
    description={A formalized and simplified representation of the processes identified in the perceptual model. This model involves creating hypotheses and adopting assumptions to abstract the complex processes of reality in a tangible and objective way, often using mathematical formulations}
}

% procedural model
\newglossaryentry{proced-model}
{
    name=procedural model,
    description={A practical representation of a conceptual model in a computer program, where the equations and concepts of the conceptual model are translated into code, allowing simulations and predictions of flows and levels based on input data}
}

% homology
\newglossaryentry{homology}
{
    name=homology,
    description={A formal analogy made in modeling, which is an equivalence between the mathematical structures of the target system and the model}
}

% leverage points
\newglossaryentry{leverage-pts}
{
    name=leverage points,
    description={In systems dynamics, leverage points are strategic locations within a complex system where a small change in one aspect can lead to significant changes in the system's behavior, making them crucial for effective interventions and systemic changes}
}

% input data
\newglossaryentry{input-data}
{
    name=input data,
    description={Input data are the information or values provided to a model for processing or analysis, serving as the basis for generating results or simulating the behavior of the system under study}
}

% congested output problem
\newglossaryentry{problem-congest-outputs}
{
    name=congested output problem,
    description={The difficulty of applying balance equations using the Euler method when outflows feed into other compartments that may eventually become saturated. One solution is to compute both the maximum and potential flows before defining the actual flow}
}

% simultaneous depletion problem
\newglossaryentry{problem-simult-deplet}
{
    name=simultaneous depletion problem,
    description={The difficulty of applying balance equations using the Euler method when multiple outflows drain the level of a compartment. A solution is to proportionally allocate between the individual flows if the total outflow exceeds the level in the simulated time step}
}

% actual flow
\newglossaryentry{flux-actual}
{
    name=actual flow,
    description={The flow that actually influences the level of a compartment in the simulation of a dynamic system}
}

% potential flow
\newglossaryentry{flux-pot}
{
    name=potential flow,
    description={A calculated inflow or outflow that potentially alters the level of a compartment in the simulation of a dynamic system. The flow needs to be confronted with the imposed physical constraints (usually conservation and non-negativity)}
}

% maximum flow
\newglossaryentry{flux-max}
{
    name=maximum flow,
    description={The highest possible flow defined by the physical constraint of a given compartment}
}

% storage deficit
\newglossaryentry{storage-deficit}
{
    name=storage deficit,
    description={The available storage of a compartment that has a maximum capacity}
}

% regularization effect
\newglossaryentry{regular-effect}
{
    name=regularization effect,
    description={The stabilization of water flow over time, minimizing extreme variations and ensuring availability for longer periods}
}

% exponential growth curve
\newglossaryentry{curve-exp-grw}
{
    name=exponential growth curve,
    description={A graph that represents the accelerated increase in a compartment's level over time, usually resulting from the dominance of inflows over outflows, associated with the presence of reinforcing loops (positive feedback) on these flows}
}

% exponential decay curve
\newglossaryentry{curve-exp-dec}
{
    name=exponential decay curve,
    description={A graph that represents a rapid drop in a compartment's level over time, usually resulting from the dominance of outflows over inflows, associated with the presence of reinforcing loops (positive feedback) on these flows}
}

% reinforcing loop
\newglossaryentry{loop-rei}
{
    name=reinforcing loop,
    description={Feedback that acts on both inflows and outflows, increasing the value of this flow, which can result in exponential behaviors (growth or decay)}
}

% balancing loop
\newglossaryentry{loop-bal}
{
    name=balancing loop,
    description={Feedback that acts on both inflows and outflows, reducing the value of this flow, leading to a situation that tends towards a state of equilibrium with or without oscillations}
}

% logistic curve
\newglossaryentry{curve-log}
{
    name=logistic curve,
    description={A graph that represents the alternation between the dominance of reinforcing loops and balancing loops, showing initially rapid growth (or decay) that later stabilizes at a plateau due to balancing effects}
}

% overshoot and collapse curve
\newglossaryentry{curve-oac}
{
    name=overshoot and collapse curve,
    description={A typical graph of systems with two main compartments, where one compartment is drained by the other, producing a pattern of accelerated growth followed by an abrupt drop in levels when resources are exhausted}
}

% reproducibility problem
\newglossaryentry{problem-reprod}
{
    name=reproducibility problem,
    description={A typical problem of dynamic systems models pointed out by John Sterman, where models are difficult to use by anyone other than their developers}
}

% sensitivity analysis
\newglossaryentry{sal}
{
    name=sensitivity analysis,
    description={A model diagnostic technique that seeks to understand how the modeled system responds to changes in its elements, such as inflows and parameter values}
}

% parametric space
\newglossaryentry{space-params}
{
    name=parametric space,
    description={The set of all possible combinations of a model's parameters, used to explore and analyze how different parameter values affect the system's behavior and results. In general, the parametric space has N dimensions, where N is the number of parameters}
}

% dimensionality problem
\newglossaryentry{problem-dimens}
{
    name=dimensionality problem,
    description={The difficulty of exploring high-dimensional parametric spaces, requiring exorbitant computational resources to execute simulations in a reasonable time frame}
}

% exhaustive sampling
\newglossaryentry{brute-force}
{
    name=exhaustive sampling,
    description={Also known as brute force, it is a strategy for sampling the parametric space by enumerating all possibilities after uniform discretization}
}

% Latin Hypercube Sampling
\newglossaryentry{lhs}
{
    name=Latin Hypercube Sampling,
    description={A statistical sampling strategy used to generate sets of sample points in a high-dimensional space efficiently, ensuring that each dimension is equally represented in all parts of its interval, improving the coverage and representativeness of samples compared to simple random sampling methods}
}

% model diagnostics
\newglossaryentry{model-diags}
{
    name=model diagnostics,
    description={A broad set of techniques applied to assess the adequacy of a model in various aspects. In systems dynamics, John Sterman lists the following diagnostics: boundary adequacy; structural adequacy; dimensional consistency; parameter distribution; comparative studies; integration error; extreme conditions; sensitivity analysis; anomalous behaviors; empirical adequacy; surprises; and positive impacts}
}

% objective function
\newglossaryentry{obj-func}
{
    name=objective function,
    description={A mathematical expression that defines the variable (or set of variables) to be maximized or minimized in an optimization problem}
}

% hydrological response
\newglossaryentry{hydro-response}
{
    name=hydrological response,
    description={The way a watershed's outflows (runoff) manifest in response to inflows (rainfall). Typically, there is a clear separation between rapid responses (rises) and slow responses (recessions)}
}

% linear reservoir
\newglossaryentry{linear-reserv}
{
    name=linear reservoir,
    description={A compartment that exhibits an outflow directly proportional to its level: $Q_t = S_t/k$, where $Q$ is the reservoir's outflow at time $t$; $S$ is the reservoir's level at time $t$, and $k$ is the reservoir's mean residence time}
}

% hydrology
\newglossaryentry{hydrology}
{
    name=Hydrology,
    description={The natural science that studies the hydrological cycle in its terrestrial phase on continents}
}

% engineering bias
\newglossaryentry{bias-engineer}
{
    name=engineering bias,
    description={A formative trend in Hydrology that seeks knowledge aimed at solving practical societal problems}
}

% science-management duality
\newglossaryentry{dual-sci-mgmt}
{
    name=science-management duality,
    description={A characteristic of Hydrology, existing at the interface between theoretical investigation of nature and practical solutions for social, environmental, and economic issues}
}

% fluvialist bias
\newglossaryentry{bias-fluvial}
{
    name=fluvialist bias,
    description={A hegemonic trend in Hydrology that focuses on essentially hydraulic problems at the watershed scale, such as flow propagation in river channels and flooding of adjacent plains}
}

% zero-order basin
\newglossaryentry{zero-basin}
{
    name=zero-order basin,
    description={Regions on slopes and higher grounds of the landscape where precipitation interacts with vegetation, soil, and rocks, producing the hydrological responses observed downstream in rivers. Also known as \textbf{slope basin}}
}

% infiltration
\newglossaryentry{infiltration}
{
    name=infiltration $f$,
    description={The flow of surface water into the soil matrix}
}

% effective rainfall
\newglossaryentry{ground-rain}
{
    name=effective rainfall $p_{\text{s}}$,
    description={The flow of rainfall that reaches the soil surface after the vegetation canopy is saturated}
}

% interception capacity
\newglossaryentry{intercep-capacity}
{
    name=interception capacity $c_{\text{max}}$,
    description={The maximum storage level of water in the vegetation canopy before effective rainfall is produced}
}

% vegetation canopy
\newglossaryentry{canopy}
{
    name=vegetation canopy $\textbf{C}$,
    description={The leaves and branches of plants that act as a compartment or reservoir that stores water through surface tension before the rain reaches the soil surface}
}

% interception
\newglossaryentry{interception}
{
    name=interception,
    description={The initial flow that fills the vegetation canopy with rainwater}
}

% vadose zone
\newglossaryentry{unsat-zone}
{
    name=vadose zone $\textbf{V}$,
    description={A porous matrix of solid soil minerals that stores water in films held by surface tension. Also known as the unsaturated zone}
}

% field capacity
\newglossaryentry{fmc}
{
    name=field capacity $v_{\text{max}}$,
    description={The maximum storage level of capillary water in the vadose zone}
}

% capillary deficit
\newglossaryentry{fmd}
{
    name=capillary deficit $\textbf{D}_\text{v}$,
    description={The potential storage capacity of capillary water in the vadose zone}
}

% phreatic zone
\newglossaryentry{sat-zone}
{
    name=phreatic zone $\textbf{G}$,
    description={A porous matrix of soil and rock that stores water in an unconfined aquifer under atmospheric pressure. Also known as the saturated zone}
}

% gravitational deficit
\newglossaryentry{dgrav}
{
    name=gravitational deficit $\textbf{D}$,
    description={The potential storage capacity of gravitational water in the vadose zone or the effective depth of the water table}
}

% recharge
\newglossaryentry{qv}
{
    name=recharge $q_{\text{v}}$,
    description={The vertical water flow (percolation) that transfers water from the vadose zone to the phreatic zone. Also known as ultimate percolation}
}

% infiltration capacity
\newglossaryentry{infmax}
{
    name=infiltration capacity $f_{\text{max}}$,
    description={The maximum potential infiltration flow determined by the characteristics of the soil surface}
}

% hydraulic conductivity
\newglossaryentry{cond-hyd}
{
    name=hydraulic conductivity $K$,
    description={The maximum potential percolation flow of water in the phreatic zone of an aquifer}
}

% excess rainfall
\newglossaryentry{ex-rain}
{
    name=excess rainfall $p_{x}$,
    description={The excess rainfall that exceeds the soil's infiltration capacity}
}

% surface retention capacity
\newglossaryentry{sfmax}
{
    name=surface retention capacity $s_{\text{max}}$,
    description={The maximum storage level in surface depressions}
}

% surface runoff
\newglossaryentry{sf-runoff}
{
    name=surface runoff $q_{si}$,
    description={The generalized surface flow produced due to the soil's relatively lower infiltration capacity compared to the effective rainfall}
}

% antecedent moisture conditions
\newglossaryentry{amc}
{
    name=antecedent moisture conditions,
    description={The set of initial conditions of the system on the eve of a rainfall event}
}

% recession curve
\newglossaryentry{deple-curve}
{
    name=recession curve,
    description={The drainage curve of the phreatic zone displayed on a river's hydrograph during a drought (baseflow)}
}

% baseflow
\newglossaryentry{ground-flow}
{
    name=baseflow $Q_{\text{g}}$,
    description={The outflow or drainage of the phreatic zone, usually considered a slow response of the watershed}
}

% aquifer detention time
\newglossaryentry{g-coef}
{
    name=aquifer detention time $g$,
    description={A characteristic parameter of the aquifer, assuming the phreatic zone behaves as a linear reservoir}
}

% hydrological cycle
\newglossaryentry{hydro_cicle}
{
    name=hydrological cycle,
    description={The circular flow of water on Planet Earth, energized by solar radiation. Evaporation transfers surface water to the atmosphere, and this water returns to the surface in the form of precipitation as rain, dew, and snow}
}

% Infiltration Age
\newglossaryentry{age_inf}
{
    name=Infiltration Age,
    description={The period between 1930 and 1970 when the scientific community in Hydrology operated under the normality of the Hortonian paradigm, which established infiltration as the key process to explain the alternation between rises and recessions in rivers}
}

% runoff by excess of infiltration
\newglossaryentry{rie}
{
    name=runoff by excess of infiltration,
    description={The generalized surface flow produced due to the soil's relatively lower infiltration capacity compared to the effective rainfall. A synonym for \textbf{surface runoff}}
}

% Curve Number Method
\newglossaryentry{cn_method}
{
    name=Curve Number Method (\acrfull{cn}),
    description={An empirical method for estimating the water balance in watersheds developed by the Soil Conservation Service in the 1950s}
}

% exfiltration
\newglossaryentry{subsur_flow}
{
    name=exfiltration $q_{\text{ss}}$,
    description={Flow from ephemeral springs located at the base of slopes, resulting from the rapid passage of water through organic horizons with large amounts of macropores. Common in forests. Considered a fast response}
}

% direct rainfall
\newglossaryentry{rse}
{
    name=direct rainfall $q_{\text{se}}$,
    description={The surface flow produced due to rainfall on saturated soil. It can be generalized but occurs more frequently in riparian wetland areas. Common on convergent slopes, forcing the water table to surface. Considered a fast response}
}

% translational flow
\newglossaryentry{trans_flow}
{
    name=translational flow $Q_{\text{gt}}$,
    description={The water flow from the phreatic zone produced by the sudden pressurization of capillary fringes in riparian areas, where the water table is near the surface. Considered a fast response}
}

% macroporosity
\newglossaryentry{macropor}
{
    name=macroporosity,
    description={A network of pores that stores and conducts water in much greater proportion than if considering the apparent porosity of the soil matrix. Higher in structured organic soils with the presence of fauna and flora (bioturbation). Common in weathered rocks, with the presence of fractures and other geological structures}
}

% Differentiation Age
\newglossaryentry{age_diff}
{
    name=Differentiation Age,
    description={The period from 1970 to the present in which the scientific community in Hydrology seeks to establish how different environments exhibit distinct response mechanisms, depending on climatic, topographic, and land use conditions}
}

% perennial springs
\newglossaryentry{nasc_perenes}
{
    name=perennial springs,
    description={Points and patches in the landscape where the main unconfined aquifer of a watershed surfaces, contributing to the baseflow in the slow response of rivers (recessions)}
}

% ephemeral springs
\newglossaryentry{nasc_efemeras}
{
    name=ephemeral springs,
    description={Points and patches in the landscape where small perched aquifers emerge, formed in more superficial organic soil horizons, contributing to exfiltration in the fast response of rivers (rises)}
}

% gravitational water
\newglossaryentry{water_gravit}
{
    name=gravitational water $\textbf{V}_{\text{g}}$,
    description={Water stored in the vadose zone that is free to percolate vertically to the phreatic zone under the influence of gravity (recharge)}
}

% capillary water
\newglossaryentry{water_capilar}
{
    name=capillary water $\textbf{V}_{\text{c}}$,
    description={Water stored in the vadose zone, held by the cohesive forces (surface tension) of soil particles. This water is more accessible for plant transpiration than for recharge to the phreatic zone}
}

% permeability transitions
\newglossaryentry{perm_trans}
{
    name=permeability transitions,
    description={Changes in effective hydraulic conductivity observed between different soil horizons. Generally, hydraulic conductivity increases in more superficial horizons due to macroporosity}
}

% organic horizon
\newglossaryentry{o-horizon}
{
    name=organic horizon $\textbf{O}$,
    description={A general term for the upper layer of soil with greater macroporosity due to the action of soil fauna and flora}
}

% riparian wetlands
\newglossaryentry{sat_areas}
{
    name=riparian wetlands,
    description={Zones in valley bottoms, near watercourses, where the water table frequently surfaces}
}

% variable source area
\newglossaryentry{vsa}
{
    name=variable source area,
    description={The phenomenon of expansion and contraction of riparian wetland areas that produces a variable contribution of runoff from direct rainfall. The variation of the source area can occur during a rainfall event or over the course of seasons}
}

% capillary fringe
\newglossaryentry{fringe}
{
    name=capillary fringe,
    description={A region in the vadose zone with zero capillary deficit due to its proximity to the phreatic zone, from which water is suctioned}
}

% convergent slopes
\newglossaryentry{hollows}
{
    name=convergent slopes,
    description={Slopes where surface and subsurface drainage converge toward a single region, generally producing riparian wetland areas}
}

% divergent slopes
\newglossaryentry{spurs}
{
    name=divergent slopes,
    description={Slopes where surface and subsurface drainage spread out in different directions, preventing the formation of riparian wetland areas}
}

% isotopic signature
\newglossaryentry{iso_sign}
{
    name=isotopic signature,
    description={The concentration of isotopes that allows identifying or tracing the origin of water from its thermal fractionation process in the atmosphere}
}

% geochemical signature
\newglossaryentry{geo-sign}
{
    name=geochemical signature,
    description={The concentration of solutes that allows identifying or tracing the origin of water from its diffusion process in the soil and rocks}
}

% thermal fractionation
\newglossaryentry{term_frac}
{
    name=thermal fractionation,
    description={A change in the concentration of isotopes caused by phase changes of water (evaporation and condensation). The water from a given rain event has a different isotopic signature from ocean water (and other rain events) due to its trajectory of thermal fractionation}
}

% bedrock topography
\newglossaryentry{bedrock_topo}
{
    name=bedrock topography,
    description={Irregularities in the bedrock that establish the relatively impermeable bottom of the unconfined aquifer in the phreatic zone. These irregularities can create pockets or stagnant zones in the aquifer}
}

% old water paradox
\newglossaryentry{old_water_paradox}
{
    name=old water paradox,
    description={The difficulty in explaining the rapid mobilization and high prevalence of old water in rivers after precipitation of new water, as well as the diversity of geochemical signatures of old water}
}

% hydro-geochemical compartmentalization
\newglossaryentry{geo_hydro_sep}
{
    name=hydro-geochemical compartmentalization,
    description={The concept of separation at multiple scales of water residence times in the pores of the phreatic zone, creating a diversity of geochemical water signatures}
}

% hydro-ecological compartmentalization
\newglossaryentry{eco_hydro_sep}
{
    name=hydro-ecological compartmentalization,
    description={The concept of separating water in the vadose zone between water widely absorbed by plant rootlets, occupying soil micropores, and water that is rapidly drained (exfiltration and recharge) through soil macropores}
}

% two-world hypothesis
\newglossaryentry{two_world}
{
    name=two-world hypothesis,
    description={A testable hypothesis about hydro-ecological compartmentalization proposed by Jeffrey McDonnell. One world of water would be the water for plants (green water), and the other world would be the water for rivers (blue water)}
}

% Unit Hydrograph
\newglossaryentry{unit_hydro}
{
    name=Unit Hydrograph,
    description={The minimal linear response of a watershed. Complex responses can be constructed from the unit hydrograph through convolution}
}

% time of concentration
\newglossaryentry{time_conc}
{
    name=time of concentration,
    description={An effective response time parameter used to determine the unit hydrograph}
}

% data-driven models
\newglossaryentry{models_data}
{
    name=data-driven models,
    description={Hydrological models that focus on observational data analysis and forecasting, using techniques such as artificial neural networks. They aim to maximize predictive capability (\gls{pred_cap}), but may compromise the explanatory capability (\gls{explan_cap}) of hydrological processes, often treating the watershed as a black box}
}

% process-driven models
\newglossaryentry{models_process}
{
    name=process-driven models,
    description={Hydrological models that represent the physical processes occurring in a watershed. They allow the simulation of watershed behavior even without empirical observations, offering a theoretical basis for runoff generation and enabling \gls{infer-dedu} of hydrological processes}
}

% predictive capability
\newglossaryentry{pred_cap}
{
    name=predictive capability,
    description={The ability of hydrological models to predict runoff behavior based on input data, such as precipitation. It refers to the accuracy and reliability of the model's forecasts}
}

% explanatory capability
\newglossaryentry{explan_cap}
{
    name=explanatory capability,
    description={The ability of hydrological models to provide an understanding of the physical processes that generate runoff. This includes the ability to describe how and why observed hydrological phenomena occur, beyond merely predicting outcomes}
}

% scale problem
\newglossaryentry{scale_problem}
{
    name=scale problem,
    description={The difficulties that arise when the scale represented by the hydrological model differs from the scale of empirical observations. This discrepancy can introduce errors into the model's results, making them incommensurable or incompatible with observed evidence}
}

% compensating effects
\newglossaryentry{comp-eff}
{
    name=compensating effects,
    description={Internal discrepancies in hydrological models caused by mass balance constraints in model compartments. These effects can lead to unrealistic results, as mass balances can mask underlying physical processes, compromising the accuracy of simulations}
}

% physically based models
\newglossaryentry{models-phys}
{
    name=physically based models,
    description={Hydrological models based on fundamental physical laws, such as the conservation of mass, momentum, and energy. They differ from systemic models by using continuous representations of vector fields to simulate hydrological processes, providing a more detailed and theoretically consistent description of watershed behavior}
}

% Darcy's Law
\newglossaryentry{darcy-law}
{
    name=Darcy's Law,
    description={A principle that describes water flow through porous media. It establishes that the flow is directly proportional to the cross-sectional area of the conduit and the difference in hydrostatic potential, and inversely proportional to the length of the conduit. Mathematically, it is expressed as $\textbf{u} = -K \nabla \Phi$, where $\textbf{u}$ is the Darcy velocity, $K$ is the hydraulic conductivity, and $\nabla \Phi$ is the hydrostatic potential gradient}
}

% global scale
\newglossaryentry{glob-scale}
{
    name=global scale,
    description={A level of analysis that assesses the aggregated and integrated behavior of the hydrological system as a whole, considering all its parts and interactions from a macroscopic perspective}
}

% local scale
\newglossaryentry{loc-scale}
{
    name=local scale,
    description={A level of analysis that focuses on infinitesimal elements of the soil, allowing for a detailed and accurate representation of hydrological processes in small areas or units, facilitating the description of hydrological phenomena from a microscopic perspective}
}

% Kalinin-Miyukov-Nash model
\newglossaryentry{nash-model}
{
    name=Kalinin-Miyukov-Nash model,
    description={A hydrological model representing the response of a watershed as a network of reservoirs arranged in series, known as a cascade. It uses a Gamma distribution to parameterize the hydrograph, incorporating parameters such as the hydrograph volume ($\nu$), the effective number of reservoirs ($n$), and the mean residence time of the reservoirs ($k$). Developed independently by Kalinin \& Miyukov (1957) and Nash (1958)}
}

% finite difference method
\newglossaryentry{finite-diff-method}
{
    name=finite difference method,
    description={A numerical technique used to solve partial differential equations, employed in the simulation of transient flows in porous media. This method discretizes the spatial domain into a regular grid, allowing the approximation of the derivatives involved in the physical equations}
}

% computational grid
\newglossaryentry{comp-grid}
{
    name=computational grid,
    description={A spatial discretization structure used in numerical methods to simulate hydrological processes. It consists of dividing the spatial domain into elements or cells, which can be regular or irregular, facilitating the application of techniques such as the finite difference method or finite element method to solve the equations that describe physical processes}
}

% finite element method
\newglossaryentry{finite-elem-method}
{
    name=finite element method,
    description={A numerical technique used to solve partial differential equations in complex domains, employing an irregular computational grid. It allows for more flexible representation of the domain's geometry and is particularly useful in simulating transient flows in porous media with varied geometries}
}

% scalability premise
\newglossaryentry{scale-prem}
{
    name=scalability premise,
    description={An inherent assumption in hydrological models that the physical processes represented can be consistently applied across different spatial scales. This premise idealizes that the same principles and physical relationships hold true regardless of scale, which often does not correspond to reality due to variations and complexities in hydrological processes at different scales}
}

% predictive models
\newglossaryentry{pred-models}
{
    name=predictive models,
    description={Hydrological models used to solve specific practical problems, focusing on predicting hydrological events under given conditions. These models apply parameters conditioned by empirical observations in specific temporal and spatial contexts to generate forecasts in different situations}
}

% scaling
\newglossaryentry{scalab}
{
    name=scaling,
    description={The process of transferring information between different spatial and temporal scales in hydrological modeling, involving the adaptation of data and parameters from one scale to another to ensure model consistency and accuracy}
}

% natural scale
\newglossaryentry{nat-scale}
{
    name=natural scale,
    description={A level of analysis referring to the actual characteristic speeds exhibited by hydrological processes in nature, including the lifespan of intermittent events, annual event periods, and trends in long-term stochastic processes}
}

% observational scale
\newglossaryentry{obs-scale}
{
    name=observational scale,
    description={A level of analysis related to the scale of empirical observations in hydrological modeling, including aspects such as data extent, sampling resolution, and sampling integration intervals}
}

% conceptual scale
\newglossaryentry{model-scale}
{
    name=conceptual scale,
    description={A level of analysis that serves as a bridge between the natural scale of hydrological processes and the observational scale of empirical evidence, representing the processes and interactions in the model in a structured way}
}

% upscaling
\newglossaryentry{upscaling}
{
    name=upscaling,
    description={The process of transferring information from smaller scales to larger scales in hydrological modeling, often performed through averaging or summation over a specific spatial or temporal extent}
}

% downscaling
\newglossaryentry{downscaling}
{
    name=downscaling,
    description={The process of transferring information from larger scales to smaller scales in hydrological modeling, which generally involves non-trivial methods and auxiliary hypotheses that consider the heterogeneity of hydrological processes across scales, including the use of co-variables (indicators)}
}

% hydrological response units
\newglossaryentry{urh}
{
    name=hydrological response units,
    description={Spatial segments or blocks in a watershed that represent hydrologically homogeneous regions in terms of hydrological response, facilitating semi-distributed modeling by grouping areas with similar hydrological behavior}
}

% regionalization
\newglossaryentry{regionalization}
{
    name=regionalization,
    description={The process of adapting and applying hydrological models developed for a specific region to other regions with different characteristics, involving the generalization of model parameters and processes to fit new geographical and hydrological conditions}
}

% distributed models
\newglossaryentry{models-ditrib}
{
    name=distributed models,
    description={Hydrological models that represent hydrological processes on a detailed spatial grid, allowing for the simulation of local variations in parameters and flows. These models can capture spatial heterogeneity and provide more accurate results at local scales, though they generally require greater computational power}
}

% scaling function
\newglossaryentry{scaling-func}
{
    name=scaling function,
    description={A mathematical function that defines how information is transferred between different scales in hydrological modeling, determining the form of aggregation or distribution of parameters and variables to maintain model consistency and accuracy}
}

% spatial heterogeneity
\newglossaryentry{hetspatial}
{
    name=spatial heterogeneity,
    description={Variability or diversity in the spatial distribution of hydrological characteristics, such as soils, vegetation cover, topography, and hydrodynamic properties. Spatial heterogeneity significantly influences the hydrological response of a watershed, requiring models to accurately represent this diversity to simulate runoff and infiltration processes}
}

% distribution function
\newglossaryentry{downscaling2}
{
    name=distribution function,
    description={A method or function used in hydrological modeling to distribute or adjust parameters or variables from a larger scale to a smaller scale, or vice versa. This function helps transfer information between different levels of spatial detail, ensuring that local characteristics are adequately represented in the model}
}

% hydrological similarity
\newglossaryentry{hydro-simi}
{
    name=hydrological similarity,
    description={A condition in which different regions or hydrological units exhibit similar hydrological behavior, allowing them to be grouped or treated uniformly in hydrological models. Hydrological similarity is used to simplify distributed modeling by grouping areas with homogeneous hydrological responses}
}

% semi-distributed models
\newglossaryentry{models-semid}
{
    name=semi-distributed models,
    description={Hydrological models that use an intermediate approach between fully distributed models and aggregated models. These models divide the watershed into hydrological response units, allowing for a more detailed representation than aggregated models but with less computational complexity than distributed models}
}

% saturation index
\newglossaryentry{tsi}
{
    name=saturation index,
    description={A topographic indicator used in hydrological modeling to represent soil saturation in a given area. The saturation index relates topographic characteristics such as slope and drainage area to determine the propensity of an area to become saturated during rainfall events, and it is fundamental for the local distribution of soil water deficit in hydrological models}
}

% Topographic Wetness Index (TWI)
\newglossaryentry{twi}
{
    name=Topographic Wetness Index (\texttt{TWI}),
    description={A topographic index used to estimate soil moisture based on terrain slope and drainage area. It is calculated by the formula $\text{T}_i = \ln{(\alpha_{i}/\tan \beta_{i})}$, where $\alpha_{i}$ is the local drainage area per unit contour length, and $\beta_{i}$ is the local terrain slope. \texttt{TWI} helps identify areas prone to saturation and runoff, and is widely used in hydrological models such as \texttt{TOPMODEL}}
}

% hydraulic transmissivity
\newglossaryentry{trans-hyd}
{
    name=hydraulic transmissivity,
    description={A soil property that represents the capacity of the porous medium to transmit water per unit of width and depth. Equivalent to hydraulic conductivity per unit of lateral contour}
}

% digital elevation model (DEM)
\newglossaryentry{dem}
{
    name=digital elevation model (\texttt{DEM}),
    description={A digital representation of the Earth's surface that captures the topography of a specific area, used in geoprocessing and hydrological modeling to derive characteristics such as slope, drainage area, and topographic wetness indices}
}

% Height Above Nearest Drainage (HAND)
\newglossaryentry{hand}
{
    name=Height Above Nearest Drainage (\texttt{HAND}),
    description={A topographic index representing the height of a point relative to the nearest drainage channel. It is calculated as the difference between the local altitude and the altitude of the nearest drainage point. \texttt{HAND} is used to map wetlands and identify flood risks, making it a valuable tool for hydrological modeling and geoprocessing by representing the topography in relation to drainage networks}
}

% separating surface
\newglossaryentry{sepsurf}
{
    name=separating surface,
    description={A surface created by the vertical permeability transition between soil horizons, separating vertical flow into a lateral component. The concept is generalized by the theory of connectivity}
}

% activation threshold
\newglossaryentry{activ-threshold}
{
    name=activation threshold,
    description={The minimum level required for the connectivity of a reservoir's drainage network (channels or pores) to begin transmitting water}
}

% Connectivity Theory
\newglossaryentry{conect-theory}
{
    name=Connectivity Theory,
    description={A unifying theory of Hydrology proposed by Jeffrey McDonnell and colleagues to overcome the problems of the differentiation paradigm and address the old water paradox}
}

% saturation-activation process
\newglossaryentry{fillspill}
{
    name=saturation-activation process,
    description={A hydrological process that manifests at all scales in Connectivity Theory}
}

% activation level
\newglossaryentry{activ-level}
{
    name=activation level,
    description={Equivalent to the activation threshold}
}

% activation function
\newglossaryentry{func-activ}
{
    name=activation function,
    description={A formal hypothesis about how the activation of a reservoir develops as saturation occurs}
}

% fragmentation level
\newglossaryentry{frag-level}
{
    name=fragmentation level,
    description={A parameter that regulates the speed of activation in the saturation equation. The higher the fragmentation, the more dampened the reservoir's activation becomes. The value of the fragmentation level corresponds to the reservoir value at half the maximum activation speed}
}


\newglossaryentry{gseconomics}
{
	name=Economics,
	description={The science that studies the allocation of scarce resources among different objectives, implying choices and decisions regarding the use of these resources. Example: directing water for agriculture or energy generation}
}

\newglossaryentry{gsethics}
{
	name=Ethics,
	description={A branch of Philosophy that investigates the moral principles and values guiding human behavior, addressing what is considered right, just, and morally appropriate}
}

\newglossaryentry{existimper}
{
	name=existential imperative,
	description={The fundamental goal of living beings to maintain their own existence, involving decisions to ensure survival conditions such as obtaining food and shelter, regardless of ethical considerations}
}

\newglossaryentry{gshumanism}
{
	name=Humanism,
	description={A philosophical movement that places human beings at the center of ethical and existential considerations, in contrast to belief in supernatural deities. Humanism is seen as the basis for doctrines like Liberalism, Socialism, and Fascism, which interpret human value in distinct ways}
}

\newglossaryentry{antropoc}
{
	name=anthropocentrism,
	description={The view that places humans at the center of all decisions and objectives, focusing on human needs and interests as fundamental in any analysis or action}
}

\newglossaryentry{gnaturalism}
{
	name=Naturalism,
	description={A philosophical theory that holds that reality is composed only of natural elements, without supernatural elements, and that any explanation about the universe must come from the natural world itself}
}

\newglossaryentry{gutilitarism}
{
	name=Utilitarianism,
	description={An ethical theory that assesses the morality of actions based on their consequences, seeking to maximize well-being and reduce suffering for the greatest number of people}
}

\newglossaryentry{welbeing}
{
	name=human well-being,
	description={A condition of satisfaction and quality of life for an individual or group, which utilitarianism aims to maximize through actions that promote pleasure and reduce pain}
}

\newglossaryentry{gutility}
{
	name=utility,
	description={The capacity of an action, good, or service to satisfy the needs or desires of an individual, serving as a central measure in utilitarian theory to evaluate the moral value of choices}
}

\newglossaryentry{hedonacc}
{
	name=hedonic calculus,
	description={A method proposed by Bentham to morally evaluate an action by considering factors such as the intensity and duration of the pleasure or pain it may cause, aiming to quantify the well-being generated}
}
\newglossaryentry{adaplevel}
{
	name=Adaptation Level Theory,
	description={A theory suggesting that constant stimuli are mentally adjusted, becoming normal and requiring new variations to generate the same sensation of pleasure or displeasure}
}

\newglossaryentry{hedonmill}
{
	name=hedonic treadmill,
	description={A concept describing the tendency of humans to return to a stable level of well-being, even after positive or negative experiences, limiting sustainable increases in happiness}
}

\newglossaryentry{ultimategoal}
{
	name=ultimate goal,
	description={The fundamental purpose, from a utilitarian perspective, of maximizing well-being and minimizing suffering for the greatest possible number of individuals}
}

\newglossaryentry{microeon}
{
	name=Microeconomics,
	description={A branch of economics that studies the behavior of individual consumers and producers, exploring how their decisions influence prices and resource allocation in specific markets}
}

\newglossaryentry{macroeon}
{
	name=Macroeconomics,
	description={A branch of economics focused on analyzing large-scale economic phenomena such as economic growth, inflation, and unemployment, encompassing both national and global economies}
}

\newglossaryentry{ecoeco}
{
	name=Ecological Economics,
	description={An economic approach that integrates ecological and thermodynamic considerations, emphasizing sustainability and criticizing unlimited economic growth on a finite planet}
}

\newglossaryentry{marutilteo}
{
	name=Marginal Utility Theory,
	description={An economic theory that explains market prices based on the additional (marginal) utility gained from consuming extra units of a good, which decreases as consumption increases}
}

\newglossaryentry{gprice}
{
	name=price,
	description={The monetary amount assigned to a good or service in markets, indicating its scarcity and marginal utility}
}

\newglossaryentry{exchaval}
{
	name=exchange value,
	description={The value attributed to a good or service in a market context, reflecting what it can be traded for in monetary terms or for other goods}
}

\newglossaryentry{prindecmarut}
{
	name=principle of diminishing marginal utility,
	description={A concept indicating that the utility of consuming additional units of a good tends to decrease as total consumption increases}
}

\newglossaryentry{waterdiamonds}
{
	name=value paradox,
	description={The difference between the use value and exchange value of goods, such as water and diamonds, where essential items may have a lower market value than luxury items due to their abundance}
}

\newglossaryentry{useval}
{
	name=use value,
	description={The intrinsic value of a resource based on its ability to meet direct needs or desires, such as water for human consumption}
}

\newglossaryentry{workval}
{
	name=labour value,
	description={A Marxist concept defining the value of a good based on the labor necessary to produce it, contrasting with marginal utility theory}
}

\newglossaryentry{pareteff}
{
	name=Pareto efficiency,
	description={A situation in which resources are allocated so that no improvement is possible without worsening the condition of at least one person, indicating an efficient distribution according to marginalist theory}
}

\newglossaryentry{marketdist}
{
	name=market distortions,
	description={Factors that prevent the market from allocating resources efficiently, such as lack of information, speculation, or monopolies, which divert the ideal functioning of markets}
}

\newglossaryentry{circflowexval}
{
	name=circular flow of exchange value,
	description={A neoclassical model representing the continuous circulation of goods and services between firms and households, sustained by the exchange of monetary value, with the objective of maximizing utility}
}

\newglossaryentry{ecogrowth}
{
	name=economic growth,
	description={The expansion of value added to goods and services in an economy. In the neoclassical view, it implies an increase in utility and well-being; in ecological economics, it is limited by finite natural resources and their physical laws}
}

\newglossaryentry{ecology}
{
	name=Ecology,
	description={The science studying the relationships between organisms and their environment, analyzing population dynamics, ecosystems, and biophysical processes, essential in ecological economics for assessing environmental impacts}
}

\newglossaryentry{marktegoods}
{
	name=commodities,
	description={Goods and services with exchange value in markets, whose demand and supply are determined by economic factors such as marginal utility and scarcity}
}

\newglossaryentry{gphysicalism}
{
	name=Physicalism,
	description={An ontological doctrine postulating that everything that exists is material and can be described by the laws of physics, a worldview aligned with Scientific Realism and ecological economics}
}

\newglossaryentry{consproblem}
{
	name=hard problem of consciousness,
	description={The philosophical difficulty of explaining subjective experience (consciousness) solely through physical and synaptic processes, one of the main challenges to physicalism}
}

\newglossaryentry{willproblem}
{
	name=free will problem,
	description={A philosophical challenge questioning the existence of genuine choices in a deterministic universe; in physicalism, human decisions would be predetermined, eliminating the idea of agency}
}

\newglossaryentry{firstlaw}
{
	name=first law of thermodynamics,
	description={The principle of conservation of energy and matter, stating that neither can be created nor destroyed, only transformed. This implies the impossibility of infinite material growth}
}

\newglossaryentry{secondlaw}
{
	name=second law of thermodynamics,
	description={A law stating that entropy in an isolated system tends to increase, implying that energy degrades and useful work becomes increasingly difficult to achieve}
}

\newglossaryentry{antroposf}
{
	name=anthroposphere,
	description={The part of the biosphere encompassing the built environment inhabited by humans, where natural resources are transformed to meet societal needs}
}

\newglossaryentry{biosf}
{
	name=biosphere,
	description={The layer of Earth that encompasses all ecosystems and living beings, exchanging energy and matter with the physical environment and fundamental to ecological economics}
}

\newglossaryentry{gthoughput}
{
	name=throughput flow,
	description={A linear flow of matter and energy through the anthroposphere, consuming resources and generating waste; a central concept in ecological economics to illustrate the environmental impact of economic activities}
}

\newglossaryentry{ecolimit}
{
	name=economic limit,
	description={The point at which the incremental benefit of expanding the anthroposphere equals the incremental sacrifice of natural capital, beyond which expansion becomes uneconomic}
}

\newglossaryentry{sustdev}
{
	name=sustainable development,
	description={Balanced economic growth within the limits imposed by the biosphere, aiming to preserve natural capital for future generations}
}

\newglossaryentry{antcap}
{
	name=anthropogenic capital,
	description={Material and social structures built by humans to produce goods and services that generate well-being}
}

\newglossaryentry{natcap}
{
	name=natural capital,
	description={Natural resources and services of the biosphere that provide direct or indirect benefits for human well-being, such as water, clean air, fertile soils, and biogeochemical cycles}
}

\newglossaryentry{opcost}
{
	name=opportunity cost,
	description={The sacrifice of natural services and resources due to the expansion of anthropogenic capital over the biosphere}
}

\newglossaryentry{desutilmar}
{
	name=marginal disutility,
	description={The incremental loss of utility resulting from the reduction of natural capital as the anthroposphere expands}
}

\newglossaryentry{colaplimit}
{
	name=catastrophic limit,
	description={The point at which environmental degradation leads to nearly irreversible impacts, potentially triggering an ecological system collapse}
}

\newglossaryentry{futilelimit}
{
	name=futility limit,
	description={The point at which the expansion of anthropogenic capital generates zero or negative marginal utility, becoming directly harmful to well-being}
}

\newglossaryentry{natrec}
{
	name=natural resources,
	description={Elements of nature that can be used directly or indirectly to benefit humans, such as water, minerals, and air}
}

\newglossaryentry{recstflow}
{
	name=stock-flow resources,
	description={Resources that are materially consumed in the production process, such as oil and wood}
}

\newglossaryentry{fundserv}
{
	name=fund-service resources,
	description={Resources that provide utility without being materially consumed, such as fertile soil, infrastructure, and energy}
}

\newglossaryentry{natserv}
{
	name=natural services,
	description={Services provided by natural capital, such as pollination, climate regulation, and water purification}
}

\newglossaryentry{recrenew}
{
	name=renewable resources,
	description={Resources that can regenerate naturally within a relatively short period, such as water and wood}
}

\newglossaryentry{recnotrenew}
{
	name=non-renewable resources,
	description={Resources that form over geological timescales, such as oil and minerals}
}

\newglossaryentry{carycap}
{
	name=natural renewal rate,
	description={The maximum rate at which a renewable resource can be extracted without compromising its regenerative capacity}
}

\newglossaryentry{recmineral}
{
	name=mineral resources,
	description={Elements extracted from the Earth’s crust that are non-renewable, such as gold, copper, and oil}
}

\newglossaryentry{recbio}
{
	name=biotic resources,
	description={Biological resources that depend on living processes for their formation and renewal, such as plants and animals}
}

\newglossaryentry{recabio}
{
	name=abiotic resources,
	description={Inorganic resources from nature, such as water and minerals}
}

\newglossaryentry{recreus}
{
	name=reusable resources,
	description={Resources that can be used more than once without significant loss of quality, such as recyclable metals}
}

\newglossaryentry{recreci}
{
	name=recyclable resources,
	description={Materials that can be processed for reuse, reducing the consumption of virgin resources}
}

\newglossaryentry{deprec}
{
	name=depreciation,
	description={The deterioration of resources or anthropogenic capital due to use and the passage of time}
}

\newglossaryentry{selforg}
{
	name=self-organization,
	description={The ability of a system to organize and regulate itself without external intervention, as occurs in natural ecosystems}
}

\newglossaryentry{natservprov}
{
	name=provisioning natural services,
	description={Direct material resources provided by nature, such as food and water}
}

\newglossaryentry{natservreg}
{
	name=regulating and maintenance natural services,
	description={Natural processes that regulate environmental conditions, such as water purification and climate regulation}
}

\newglossaryentry{natservcult}
{
	name=cultural natural services,
	description={Intangible benefits provided by nature, such as recreation and cultural inspiration}
}

\newglossaryentry{probfreeacc}
{
	name=open access problem,
	description={An economic dilemma in which common resources are overexploited due to lack of regulation, leading to environmental degradation}
}

\newglossaryentry{reccom}
{
	name=common resources,
	description={Open-access resources, such as air and oceans, that everyone can use but require management to prevent overuse}
}

\newglossaryentry{external}
{
	name=externality,
	description={The impact of an economic action on the well-being of others, without that impact being reflected in the agent’s cost or benefit}
}

\newglossaryentry{rivalty}
{
	name=rivalry,
	description={A characteristic of a resource that prevents simultaneous use by multiple users}
}

\newglossaryentry{exclusvty}
{
	name=exclusivity,
	description={A characteristic that allows only one economic agent to use the resource at a given time}
}

\newglossaryentry{recrival}
{
	name=rival resources,
	description={Resources that cannot be consumed by more than one user at the same time, such as food and fuels}
}

\newglossaryentry{recexclu}
{
	name=exclusive resources,
	description={Resources whose use is guaranteed for a single user through ownership or concession}
}

\newglossaryentry{rectrad}
{
	name=tradable resources,
	description={Resources that can be exchanged in the market, facilitating allocation according to supply and demand}
}

\newglossaryentry{privprop}
{
	name=private property,
	description={A right guaranteed by the State to an individual or organization, allowing exclusive use and trade of a resource}
}

\newglossaryentry{recnonrival}
{
	name=non-rival resources,
	description={Resources that can be used simultaneously by multiple agents, such as solar radiation}
}

\newglossaryentry{reccongest}
{
	name=congestible resources,
	description={Resources that allow simultaneous use up to a certain point, such as roads or beaches, where excessive use leads to congestion}
}

\newglossaryentry{instrcc}
{
	name=command and control instruments,
	description={Regulations imposed by the State to manage the use of natural resources, such as emission standards and zoning}
}

\newglossaryentry{usepermits}
{
	name=use permit,
	description={A formal authorization issued by the State for the use of a natural resource, regulating access and the permissible amount of use}
}

\newglossaryentry{instecon}
{
	name=economic instruments,
	description={Public policy tools that use financial incentives to influence conservation behaviors and sustainable use}
}

\newglossaryentry{instrmarkt}
{
	name=market-based instruments,
	description={Policies that create markets for credits and permits for natural resources, such as carbon trading, encouraging efficiency in resource allocation}
}

\newglossaryentry{insticap}
{
	name=institutional capacity,
	description={The ability of an institution to manage resources and implement policies effectively and sustainably}
}

\newglossaryentry{costtransac}
{
	name=transaction cost,
	description={Expenses associated with negotiating a resource, including information search, enforcement, and contract implementation}
}

\newglossaryentry{econval}
{
	name=economic value,
	description={The monetary assessment of a natural service or resource, reflecting its utility to society and its importance in economic decision-making}
}

\newglossaryentry{casctmodel}
{
	name=cascade model of natural services,
	description={A model proposed to describe the flow of natural services from nature to their value for society, passing through stages such as function, service, benefit, and value}
}

\newglossaryentry{biophysistrc}
{
	name=biophysical structure,
	description={Physical components of ecosystems that provide the basis for the provision of natural services, such as forests and wetlands}
}

\newglossaryentry{ecoprocess}
{
	name=ecological processes,
	description={Natural processes that sustain environmental functions, such as nutrient cycling and water infiltration}
}

\newglossaryentry{envfunc}
{
	name=environmental functions,
	description={The capacities of an ecosystem to perform activities that can be useful to humans, such as flood regulation}
}

\newglossaryentry{econbenefit}
{
	name=economic benefit,
	description={Monetary or well-being benefits derived from natural services that contribute to human quality of life}
}

\newglossaryentry{natservgen}
{
	name=general natural services,
	description={Natural services common to different ecosystems, such as climate regulation and soil formation}
}

\newglossaryentry{natservbiome}
{
	name=biome-specific natural services,
	description={Natural services with specific characteristics of different biomes, such as tropical forests or oceans}
}

\newglossaryentry{natservclim}
{
	name=climate regulation natural service,
	description={An ecosystem service that contributes to climate regulation through processes such as carbon sequestration and plant transpiration}
}

\newglossaryentry{natservsoil}
{
	name=soil natural services,
	description={Services provided by soil, including plant support, water retention, and nutrient cycling}
}

\newglossaryentry{natservpol}
{
	name=pollination natural service,
	description={An ecosystem service in which pollinators aid plant reproduction, essential for agricultural production}
}

\newglossaryentry{natservcontrol}
{
	name=pest control natural service,
	description={Control of pest populations by natural predators and parasites, reducing the need for pesticides}
}

\newglossaryentry{ecoserv}
{
	name=ecosystem service,
	description={Benefits that ecosystems provide for human well-being, encompassing provisioning, regulating, supporting, and cultural services}
}

\newglossaryentry{envserv}
{
	name=environmental services,
	description={Human or natural activities that support the maintenance and enhancement of ecosystem services}
}

\newglossaryentry{greeninfra}
{
	name=green infrastructure,
	description={Urban vegetation systems, such as parks and wooded areas, that offer environmental and social benefits}
}

\newglossaryentry{natservsup}
{
	name=supporting natural services,
	description={Ecosystem services essential to the functioning of ecosystems, such as oxygen production and soil formation}
}

\newglossaryentry{utilval}
{
	name=utilitarian value,
	description={Value attributed to a resource based on the utility it provides to humans, typically measured in monetary terms}
}

\newglossaryentry{intrisval}
{
	name=intrinsic value,
	description={The value of a resource or service that exists independently of its utility to humans}
}

\newglossaryentry{rightval}
{
	name=rights-based value,
	description={Non-utilitarian value attributed to natural resources based on ethical principles or inherent rights of existence, not solely on utility}
}

\newglossaryentry{gbiocentrism}
{
	name=biocentrism,
	description={An ethical approach that recognizes the intrinsic value of all forms of life, advocating for the right of species to exist}
}

\newglossaryentry{impercateg}
{
	name=categorical imperative,
	description={A Kantian moral principle guiding ethical decisions, advocating actions to be taken as if they were universal laws}
}

\newglossaryentry{biophyval}
{
	name=biophysical value,
	description={Non-utilitarian value of a natural resource calculated based on physical properties, such as energy or materials needed for its production}
}

\newglossaryentry{ecofootprint}
{
	name=ecological footprint,
	description={A measure of environmental impact that calculates the amount of natural resources consumed and the space needed to absorb waste}
}

\newglossaryentry{amcrit}
{
	name=multicriteria analysis,
	description={An evaluation method considering multiple criteria for decision-making, applicable to environmental policies and sustainability}
}

\newglossaryentry{tev}
{
	name=Total Economic Value,
	description={The sum of all utilitarian values of a resource, including use and non-use values}
}

\newglossaryentry{nonuseval}
{
	name=non-use value,
	description={Utilitarian value attributed to a natural resource or service without direct use, such as preservation value}
}

\newglossaryentry{legval}
{
	name=legacy value,
	description={Utilitarian value attributed to the preservation of resources for the benefit of future generations}
}

\newglossaryentry{altrval}
{
	name=altruistic value,
	description={Utilitarian value reflecting concern for the well-being of other members of the current generation}
}

\newglossaryentry{existval}
{
	name=existence value,
	description={Utilitarian satisfaction derived simply from knowing that a species or ecosystem continues to exist}
}

\newglossaryentry{direcuseval}
{
	name=direct use value,
	description={Utilitarian value obtained from the direct use of a resource, such as timber extraction or park visitation}
}

\newglossaryentry{indirectuseval}
{
	name=indirect use value,
	description={Utilitarian value indirectly obtained from a resource, such as the benefit of climate regulation by a forest}
}

\newglossaryentry{useconsum}
{
	name=consumptive,
	description={Use of resources that results in their irreversible transformation or consumption, as in fuel burning}
}

\newglossaryentry{usenonconsun}
{
	name=non-consumptive,
	description={Use of resources without direct transformation or consumption, such as recreational activities in natural areas}
}

\newglossaryentry{futuseval}
{
	name=future use value,
	description={Utilitarian value associated with the potential future use of a natural resource, such as biodiversity for future scientific discoveries}
}

\newglossaryentry{optionval}
{
	name=option value,
	description={Utilitarian value associated with preserving a resource for future use, considering the uncertainty about its potential utilities}
}
\newglossaryentry{tippingpoint}
{
	name=point of no return,
	description={A critical threshold where an environmental change becomes irreversible, leading to a significant transformation in the ecosystem}
}

\newglossaryentry{riskaversv}
{
	name=risk aversion,
	description={The tendency to avoid changes that could result in significant environmental or economic losses}
}

\newglossaryentry{precauprinci}
{
	name=precautionary principle,
	description={A principle recommending preventive measures in the face of uncertainty about environmental risks}
}

\newglossaryentry{valuadm}
{
	name=direct market valuation,
	description={A method using actual market prices to estimate the utilitarian value of natural resources and services}
}

\newglossaryentry{valuarevpref}
{
	name=revealed preferences valuation,
	description={A method that deduces the utilitarian value of a resource based on behavior in indirectly related markets}
}

\newglossaryentry{valuastatpref}
{
	name=stated preference valuation,
	description={A method that estimates the utilitarian value of resources through surveys with individuals in simulated contexts}
}

\newglossaryentry{valuaprice}
{
	name=price-based valuation,
	description={A method that calculates the utilitarian value of a resource based on market prices, adjusting for distortions when necessary}
}

\newglossaryentry{valuacost}
{
	name=cost-based valuation,
	description={A method that assesses the utilitarian value of a service or resource based on the costs required for its replacement or maintenance}
}

\newglossaryentry{valuafuncprod}
{
	name=production function-based valuation,
	description={A method that links the performance of natural services to the production of goods in markets}
}

\newglossaryentry{valuaavoid}
{
	name=avoided cost method,
	description={A valuation method that considers the costs avoided by the presence of a natural service, such as flood protection}
}

\newglossaryentry{valuarepo}
{
	name=replacement cost method,
	description={A method that calculates the utilitarian value of a service based on the cost of replacing it with artificial alternatives}
}

\newglossaryentry{valuarest}
{
	name=mitigation or restoration cost method,
	description={A method that assesses the utilitarian value of a natural service based on the costs required to restore it after degradation}
}

\newglossaryentry{valuatravcost}
{
	name=travel cost method,
	description={A method that estimates the utilitarian value of natural areas based on the costs incurred by visitors to access these areas}
}

\newglossaryentry{valuahedonpric}
{
	name=hedonic pricing method,
	description={A method that uses variations in the prices of goods, such as real estate, to infer the value of environmental attributes}
}

\newglossaryentry{valuacontg}
{
	name=contingent valuation method,
	description={A method that uses surveys to obtain individuals' willingness to pay for the conservation of a natural resource}
}

\newglossaryentry{valuachoice}
{
	name=choice modeling valuation,
	description={A method that models hypothetical choices to reveal preferences for different aspects of a natural resource}
}

\newglossaryentry{valuagroups}
{
	name=group valuation,
	description={A method involving discussions and consensus among groups to assess the value of a natural service}
}
