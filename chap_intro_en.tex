\documentclass[./main_en.tex]{subfiles}
\graphicspath{{\subfix{./figs}}}

% ------------ main document ------------ 
\begin{document}
\chapter{Introduction}

% custom paragraph skip
\setlength{\parskip}{\myparskip}

--- Iporã, think about the problem first, not the tool.

\vspace{1.0cm}

\noindent This is what Professor Guilherme Marques told me on a hot December afternoon in 2017 at the Hydraulic Research Institute of the Federal University of Rio Grande do Sul, in Porto Alegre. I had been accepted into the academic master's selection in the graduate program, establishing first contacts with the researcher who would guide me for the next two years, which, with the PhD, turned into seven. \say{\textit{We’ll consider the tool later; first, we need to deeply understand the relevant research problem questions.}}, he continued, drinking coffee and explaining that choosing the method before the problem would imprison us irreparably. If necessary, he encouraged me, I would have to invent an innovative method to obtain an innovative solution. But first, I should formulate the questions well. I liked that.

\par With this guidance in mind, I went to Brasília for the World Water Forum in March 2018. At the event, the United Nations agenda (2018) \cite{un2018} motivated me to study nature-based solutions for water resource management. Upon returning to Porto Alegre, I decided to investigate how to expand green infrastructure to improve urban water security. Faced with future climate pressures and water demand, numerous questions arose: what to do? when? at what cost? is it worth it? where to invest? These questions became even more pertinent following institutional initiatives such as the National Policy for Payment for Environmental Services (2021) \cite{brasil2021} and the National Watershed Revitalization Program. The impacts of climate change, evident in 2024, also underscore the urgency of rigorously grounded adaptive strategies.

\par These questions led me to conceive and program a hydrological model, which I named \texttt{PLANS}. That is, rather than relying on generic, readily usable models, the problem forced me to design a customized tool. The first results of this journey were published in Possantti \& Marques (2022) \cite{Possantti2022a}, where we evaluated the cost-benefit and timeline of actions for scenarios in the Sinos River watershed. In this case, the main need for a flexible model was coupling it to an optimization algorithm, Dynamic Programming. As thousands of simulations were required, I configured the model as a routine in the same programming language as the algorithm.

\par From then on, we moved on to the problem of spatial allocation: \textit{where} to act? We studied the Arroio Castelhano basin in Venâncio Aires, where a pilot project for Payments for Environmental Services is underway. To generate the priority areas map, published in Possantti \textit{et al.} (2023) \cite{Possantti2023a}, I delved into the spatial representation of hydrological processes on hillslopes, which led me to philosophical reflections on modeling. To obtain results on an operational scale, a sufficiently detailed model was needed. However, physically-based models, although useful for simulating vector fields of water velocity, are computationally prohibitive and have uncertainties similar to semi-distributed models. Thus, I concluded that the principles of \texttt{TOPMODEL} were more effective in addressing the spatial allocation problem, leading to programming a new version of the \texttt{PLANS} model.

\par Over time, thinking about the problem first became a complex task. I ended up opening many doors along the problem path, in an almost endless exploration. The qualification board correctly pointed out that I was \say{moving backward}. In the case of modeling, theoretical deepening brought to light inevitable problems, such as empirical uncertainty, equifinality, and scale. Similar dilemmas arose in management, as setting priorities requires an economic principle, such as potential additionality in watersheds. Similarly, assessing the feasibility of investments in green infrastructure requires a system to value hydrological processes. Quickly, I encountered more theoretical questions, which Philosophy articulates better than Science.

\par The thesis presented is the final expression of this problem exploration. The following text provides a \textbf{synthesis} of fundamental principles and problems in the use of hydrological models for watershed conservation planning. The aim of the monograph is to establish conceptual connections to support the use of these models in this context. Unlike \textbf{analysis}, which performs conceptual breakdown, synthesis gathers distinct elements and organizes them into a coherent whole. Often, my colleagues focus on \textit{analyzing} specific scientific questions in their theses, responding to them with specialized methods and strategies. This intellectual movement is essential for local depth, but someone eventually needs to make the opposite effort, unifying analytical approaches into a coherent global view. As we will see, Science is not just about fitting pieces into a puzzle but having a vision of the final image that the completed puzzle will reveal.

\par It is worth noting that synthesis work, especially when aimed at foundational aspects, is increasingly overlooked due to academic pressure for peer-reviewed publications that demand novel results. This can lead to hydrological model usage without a critical sense of their fundamental premises. Lieke Melsen (2022) \cite{Melsen2022}, for instance, interviewed 14 water resources researchers and found that the main reason for choosing hydrological models is the influence of more experienced colleagues in the group. There is an evident efficiency in following colleagues' work, but this highlights the importance of synthesis work that revises foundations, preventing this continuity from becoming mere imitation. With the advent of Artificial Intelligence and language models, generalized imitation presents increasing challenges for truly human knowledge production. Thus, I suggest we draw inspiration from works like Keith Beven (2002) \cite{Beven2002a} in Hydrology and Herman Daly (2015) \cite{Daly2015a} in Economics, authors who seek greater coherence by making explicit underlying philosophical questions.

\par The innovative result of this synthesis, therefore, is a \textbf{map of ideas} articulated in four chapters, arranged in an ascending order of practical application and integration of systematized concepts. In addition to the chapters, a Glossary was created with the same spirit, serving as a map. Thus, the synthesis results in a structure that facilitates a comprehensive and integrated understanding of issues related to using hydrological models in integrated water resource management, especially in designing strategies to expand green infrastructure in zero-order basins. It is hoped that the scientific community will move firmly forward to articulate organized concepts on both pure and applied fronts, seeking new solutions, filling gaps, and revealing new problems and flaws that may eventually demand a complete reformulation of established foundations.

\par The first chapter emphasizes the epistemological foundations of model use, including hydrological models, which aim to convey theories about reality. The focus is on the philosophy of \textbf{instrumentalism}, which recognizes uncertainties when trying to \say{capture} reality with precise mathematical theories. The role of empirical evidence is explored from various perspectives, as well as the importance of paradigms in theory construction. The practical implication is that we can adopt a rejection criterion for hydrological models, the \textbf{encapsulation test} by the observational uncertainty band. How to conduct this test, covering from rating curve uncertainties to future scenarios, are open paths for investigation.

\par The second chapter addresses model ontology, developing a transition between theoretical and practical themes. The central point is to introduce \textbf{System Dynamics}, an approach that creates models from a network of reservoirs connected by input and output flows. Although these \say{building blocks} are simple, I illustrate how level and flow arrangements can quickly generate complex behaviors. The chapter also systematizes \textbf{model diagnostic} techniques, assessing their adequacy in various aspects. Many avenues for future research are opened, including agent-based models, exploratory modeling, sensitivity analysis, operational research, and crucially, the reproducibility problem — the difficulty of using models by others besides their creators.

\par The third chapter delves into the field of Hydrology, reviewing the evolution of this Science throughout the 20th century, marked by successive paradigm shifts precipitated by empirical evidence from experimental watersheds. The essential message is that, for watershed conservation, hydrological processes must be assessed on the hillslope scale, called \textbf{zero-order basins}. It is highlighted that hillslope runoff is part of a range of hydrological responses — fast and slow, underground and surface — that vary according to topography, soil, vegetation, climate, and season. It demonstrates that the difficulty in reconciling this complexity with hydrological models has generated debates and uncertainties, such as the problems of \textbf{equifinality} and \textbf{scale}. New promising paths emerge with connectivity theory, a recently proposed unifying paradigm.

\par Finally, the fourth chapter reaches applied watershed management, preceded by a theoretical exposition of the economic and ethical principles guiding decisions in this field. The chapter highlights the role of \textbf{Ecological Economics} in promoting a paradigm shift that sees watersheds as natural capital, providers of natural watershed services. Thus, managing watershed areas with economic instruments, such as Payment for Environmental Services (PES) schemes, is interesting to ensure water security for water users. In this context, the \texttt{PLANS} model presents itself as an adequate tool to estimate \textbf{potential additionality} at the farm level and prioritize conservation areas, considering uncertainties. New paths emerge, especially for evaluating trade-offs and synergies with other natural services and integrating the benefits of gray infrastructure.


\end{document}