\documentclass[./main.tex]{subfiles}
\graphicspath{{\subfix{./figs}}}

% ------------ main document ------------ 
\begin{document}
\chapter{Introdução}

% custom paragraph skip
\setlength{\parskip}{\myparskip}

 --- Iporã, pense primeiro no problema, não na ferramenta. 

\vspace{1.0cm} 

\noindent Foi o que o professor Guilherme Marques me disse em uma tarde quente em dezembro de 2017, no Instituto de Pesquisas Hidráulicas da Universidade Federal do Rio Grande do Sul, em Porto Alegre. Eu havia sido aprovado na seleção do mestrado acadêmico no programa de pós-graduação, e estabelecia os primeiros contatos com o pesquisador que iria me orientar nos próximos dois anos, que com o doutorado se tornaram sete. \say{\textit{A ferramenta a gente vê depois; antes precisamos entender com profundidade quais são as perguntas relevantes do problema de pesquisa.}}, ele continuou, tomando um café e explicando que o método, se eleito antes do problema, iria nos aprisionar de forma irremediável. Caso fosse necessário, ele me encorajou, eu teria que inventar um método inovador para obter uma solução inovadora. Mas antes, eu deveria formular bem as perguntas. Eu gostei disso. 

\par Com essa orientação em mente, fui a Brasília para o Fórum Mundial da Água, em março de 2018. No evento, a pauta das Nações Unidas (2018) \cite{un2018} me motivou a estudar soluções baseadas na natureza para o gerenciamento de recursos hídricos. Ao retornar a Porto Alegre, decidi investigar como expandir a infraestrutura verde para melhorar a segurança hídrica urbana. Diante das pressões do clima futuro e da demanda por água, surgiram inúmeras questões: o que fazer? quando? qual o custo? vale a pena? onde investir? Essas perguntas tornaram-se ainda mais pertinentes após inciativas institucionais como a Política Nacional de Pagamentos por Serviços Ambientais (2021) \cite{brasil2021} e o Programa Nacional de Revitalização de Bacias Hidrográficas. Os impactos das mudanças climáticas, escancarados em 2024, também reforçam a urgência de estratégias adaptativas rigorosamente fundamentadas.

\par Essas perguntas me conduziram para conceber e programar um modelo hidrológico, que eu chamei de \texttt{PLANS}. Ou seja, ao invés de depender de modelos genéricos e já prontamente utilizáveis, o caminho do problema me forçou a arquitetar uma ferramenta customizada. Os primeiros resultados dessa trajetória foram publicados em Possantti \& Marques (2022) \cite{Possantti2022a}, em que avaliamos o custo-benefício e o calendário de ações para cenários na bacia hidrológica do Rio dos Sinos. Nesse caso, a principal necessidade de um modelo maleável foi o acoplamento em um algoritmo de otimização, a Programação Dinâmica. Como milhares de simulações eram necessárias, eu configurei o modelo como uma rotina na mesma linguagem de programação do algoritmo. 

\par Daí em diante, avançamos sobre o problema de alocação espacial: \textit{onde} atuar? Estudamos a bacia do Arroio Castelhano, em Venâncio Aires, onde ocorre um projeto piloto de Pagamentos por Serviços Ambientais. Para gerar o mapa de áreas prioritárias, publicado em Possantti \textit{et al.} (2023) \cite{Possantti2023a}, me aprofundei na representação espacial dos processos hidrológicos nas encostas, o que me levou a reflexões filosóficas sobre modelagem. É que para obter resultados na escala operacional, era necessário um modelo suficientemente detalhado. Mas os modelos fisicamente embasados, embora úteis para simular campos vetoriais de velocidade da água, são proibitivos em termos computacionais e apresentam incertezas semelhantes aos modelos semi-distribuídos. Concluí, assim, que os princípios do \texttt{TOPMODEL} eram mais eficazes para abordar o problema de alocação espacial, levando à programação de uma nova versão do modelo \texttt{PLANS}.

\par Com o tempo, a orientação de pensar primeiro no problema tornou-se uma tarefa complexa. Eu acabei abrindo várias portas no caminho do problema, em uma exploração quase interminável. A banca de qualificação apontou, corretamente, que eu estava \say{avançando para trás}. No caso da modelagem, o aprofundamento teórico trouxe à tona problemas inevitáveis, como a incerteza empírica, a equifinalidade e a escala. Dilemas similares surgiram no lado da gestão, pois definir prioridades requer um princípio econômico, como o da adicionalidade potencial em bacias hidrográficas. Da mesma forma, avaliar a viabilidade de investimentos em infraestrutura verde demanda um sistema de valoração dos processos hidrológicos. Rapidamente, encontrei questões mais teóricas, que a Filosofia articula melhor do que a Ciência.

\par A tese apresentada é a expressão final dessa exploração de problemas. O texto a seguir realiza uma \textbf{síntese} dos princípios e problemas fundamentais no uso de modelos hidrológicos no planejamento da conservação de bacias hidrográficas. O objetivo da monografia é estabelecer conexões conceituais que fundamentem o uso desses modelos nesse contexto. Ao contrário da \textbf{análise}, que faz um desmembramento conceitual, a síntese reúne elementos distintos e os organiza num todo coerente. Não raro, meus colegas se ocupam em suas teses de \textit{analisar} questões científicas específicas, respondendo a elas com métodos e estratégias especializados. Esse movimento intelectual é essencial para a profundidade local, mas alguém precisa, eventualmente, fazer o esforço oposto, unificando as abordagens analíticas em uma visão global coerente. Como veremos, a Ciência não é apenas encaixar peças num quebra-cabeça, mas sim ter uma visão sobre a imagem final que o quebra-cabeça completo revelará.

\par Cabe ressaltar que o trabalho de síntese, especialmente quando voltado para fundamentações, é cada vez mais deixado de lado pela pressão acadêmica por publicações em periódicos revisados por pares, que exigem resultados inéditos. Isso pode levar a um uso de modelos hidrológicos sem um senso crítico sobre suas premissas fundamentais. Lieke Melsen (2022) \cite{Melsen2022}, por exemplo, ao entrevistar 14 pesquisadores em recursos hídricos, identificou que a principal razão para a escolha de modelos hidrológicos é a influência de colegas mais experientes no grupo. Há uma eficiência evidente em seguir o trabalho de colegas, mas isso reforça a importância de trabalhos de síntese que revisem as fundamentações, evitando que essa continuidade se torne mera imitação. Com o advento da Inteligência Artificial e dos modelos de linguagem, a imitação generalizada apresenta desafios crescentes para a produção de conhecimento verdadeiramente humano. Assim, sugiro que nos inspiremos em trabalhos como os de Keith Beven (2002) \cite{Beven2002a}, na Hidrologia, e Herman Daly (2015) \cite{Daly2015a}, na Economia, autores que buscam uma coerência maior ao explicitar questões filosóficas subjacentes.

\par O resultado inovador desta síntese, portanto, é um \textbf{mapa de ideias} que está articulado em quatro capítulos, enfileirados em uma ordem crescente de grau de aplicação prático e de integração dos conceitos sistematizados. Além dos capítulos, um Glossário foi criado com esse mesmo espírito, para cumprir a função de mapa. Assim, a síntese resulta em uma estrutura que agiliza a compreensão abrangente e integrada dos assuntos relacionados com o uso de modelos hidrológicos no gerenciamento integrado de recursos hídricos, em especial no desenho de estratégias para a expansão da infraestrutura verde em bacias de ordem zero. Com ela, deseja-se que a comunidade científica avance com firmeza para articular nas frentes puras e aplicadas os conceitos organizados, buscando novas soluções e preenchendo lacunas, mas também revelando novos problemas e falhas que eventualmente demandam uma reformulação completa dos fundamentos pré-estabelecidos.

\par O primeiro capítulo enfatiza as fundamentações epistemológicas sobre o uso de modelos, incluindo os hidrológicos, que visam veicular teorias sobre a realidade. O foco é na filosofia do \textbf{instrumentalismo}, que reconhece as incertezas ao tentarmos \say{capturar} a realidade com teorias matemáticas precisas. O papel das evidências empíricas é explorado sob várias perspectivas, assim como a importância dos paradigmas na construção de teorias. A implicação prática é que podemos adotar um critério de rejeição para modelos hidrológicos, o \textbf{teste de encapsulamento} pela banda de incerteza observacional. Como realizar esse teste, abrangendo desde a incerteza das curvas-chave até cenários futuros, são caminhos abertos para investigação.

\par O segundo capítulo aborda a ontologia de modelos, desenvolvendo uma transição entre temas teóricos e práticos. O ponto central é introduzir a \textbf{Dinâmica de Sistemas}, uma abordagem que cria modelos a partir de uma rede de reservatórios conectados por fluxos de entrada e saída. Embora esses \say{tijolos de construção} sejam simples, ilustro como arranjos de níveis e fluxos podem rapidamente gerar comportamentos complexos. O capítulo também sistematiza técnicas de \textbf{diagnóstico de modelos}, avaliando sua adequação sob diversos aspectos. Muitos caminhos para pesquisas futuras são abertos, incluindo modelos baseados em agentes, modelagem exploratória, análise de sensibilidade, pesquisa operacional e, crucialmente, o problema da reprodutibilidade, que é a dificuldade de uso de modelos por outros além de seus criadores.

\par O terceiro capítulo adentra o campo da Hidrologia, revisando a evolução dessa Ciência ao longo do século XX, marcada por sucessivas mudanças de paradigmas precipitadas por evidências empíricas de bacias experimentais. A mensagem essencial é que, para a conservação de bacias, os processos hidrológicos devem ser avaliados na escala das encostas, chamadas \textbf{bacias de ordem zero}. Destaca-se que as enxurradas nas encostas fazem parte de uma gama de respostas hidrológicas — rápidas e lentas, subterrâneas e superficiais — que variam conforme a topografia, solo, vegetação, clima e estação. Demonstra-se que a dificuldade em conciliar essa complexidade com modelos hidrológicos gerou debates e incertezas, como os problemas de \textbf{equifinalidade} e \textbf{escala}. Novos caminhos promissores surgem com a teoria da conectividade, um paradigma unificador recentemente proposto.

\par No quarto capítulo, por fim, chega-se na gestão aplicada de bacias hidrográficas, precedida por uma exposição teórica sobre os princípios econômicos e éticos que orientam as decisões nesse campo. O capítulo destaca o papel da \textbf{Economia Ecológica} em promover mudança de paradigma que vê as bacias como capital natural, fornecedoras de serviços hidrológicos naturais. Assim, a gestão de áreas de mananciais com instrumentos econômicos, como esquemas de Pagamentos por Serviços Ambientais (PSA), são interessantes para garantir a segurança hídrica dos usuários de água. Nesse contexto, o modelo \texttt{PLANS} apresenta-se como uma ferramenta adequada para estimar a \textbf{adicionalidade potencial} em nível de lotes rurais e priorizar áreas de conservação, considerando as incertezas. Novos caminhos surgem, especialmente para avaliar trade-offs e sinergias com outros serviços naturais e integrar os benefícios da infraestrutura cinza.

\end{document}